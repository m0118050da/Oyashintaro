\chapter{仮説と提案}
\label{chp:first}

3Dプリンターに実装した,氷を作るための機構について述べる.
これまでの氷の造形方法は大きく分けて2つある.大きな氷から切削して造形するもの.
もう一つが,水を少しずつたらし長時間をかけて,造形するのもがある.
どちらも取り扱いが難しく,造形するのに長時間を要してしまうのが問題だ.
また,短時間できる氷の造形として,過冷却水を使っての造形が有名である.
しかし,過冷却水の場合準備に時間がかかる上,温度変化に敏感で少しの衝撃でも凍り始めてしまうため,制御が難しい.
一般の人でも扱いが可能かつ,一般のユーザーが設計したデータにできるだけ近い形に印刷できる氷をマテリアルとした3Dプリンターの提案する.
必要な要素としては

\begin{enumerate}
  \item ある程度の精度で造形ができること. 
  \item 通常の3Dプリンターと同程度の速度で印刷ができること.
  \item 氷の定義を満たしていること.
  \item 3Dプリンターが扱える人であれば,短時間で扱えるようになること.
 \end{enumerate}

が必要であると考える.

「ある程度の精度で造形ができること.」「通常の3Dプリンターと同程度の速度で印刷ができること.」を満たすために液体窒素を使った造形方法が有効ではないかと考える.
また,「3Dプリンターが扱える人であれば,短時間で扱えるようになること.」を満たすためには,既存の3Dプリンターと同様の使用方法で使える必要があるため,世界中で使用されている3DプリントおよびスライサーソフトウェアであるUltimaker Cura で操作が可能である必要があると考える.
