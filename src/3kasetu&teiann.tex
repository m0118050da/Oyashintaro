\chapter{仮説と提案}
\label{chp:first}

\section{氷をマテリアルとした3Dプリンター}
\label{sec:paragraph}

3Dプリンターに実装した,氷を作るための機構について述べる.
これまでの氷の造形方法は大きく分けて2つある.大きな氷から切削して造形するもの.
もう一つが,水を少しずつたらし長時間をかけて,造形するのもがある.
どちらも取り扱いが難しく,造形するのに長時間を要してしまうのが問題だ.
また,短時間できる氷の造形として,過冷却水を使っての造形が有名である.
しかし,過冷却水の場合準備に時間がかかる上,温度変化に敏感で少しの衝撃でも凍り始めてしまうため,制御が難しい.
一般の人でも扱いが可能かつ,一般の3Dプリンターと同程度の速度とある程度の精度を両立した氷をマテリアルとした3Dプリンターの提案する.
必要な要素としては以下のようである.

\begin{enumerate}
  \item ある程度の精度で造形ができること. 
  \item 通常の3Dプリンターと同程度の速度で印刷ができること.
  \item 氷の定義を満たしていること.
  \item 3Dプリンターが扱える人であれば,短時間で扱えるようになること.
 \end{enumerate}

それぞれの要素について実装にするにあたり、以上のことが有効ではないかと考える.
「ある程度の精度で造形ができること.」「通常の3Dプリンターと同程度の速度で印刷ができること.」を満たすために液体窒素を使った造形方法が有効ではないかと考える.
また,他の研究では,特殊な機材を使用し,装置が高価になりがちである.液体窒素は,日本各地で手に入る上,価格も1リットルあたり300円と安価であるため,今回の研究で使用することにした.
「3Dプリンターが扱える人であれば,短時間で扱えるようになること.」を満たすためには,既存の3Dプリンターと同様の使用方法で使える必要があるため,世界中で使用されている3Dプリントおよびスライサーソフトウェアである Ultimaker Cura で操作が可能である必要があると考える.
氷の定義をしたことで、純粋な水以外でも氷の造形ができる.純粋な水を積層する場合,水の粘度が低いため,固まる前に広がってしまう.そのため造形精度が悪く,造形物のからはみ出した部分には造形ができず,オーバーハングなども造形することが難しい.
よって,水の粘度を上げることにより,上記の問題の解決や精度を向上させることができるのではと考える.
また,粘度を上げる手段として,いくつかの方法が考えられる.水に砂糖などを加え粘度を上げる方法とシャーベット状のものをマテリアルとして使用する方法だ.
シャーベット状のものを使用する場合は,温度管理が必要になるため,今回の機構では,砂糖を加え粘度を高めたものをマテリアルとして使用する.

