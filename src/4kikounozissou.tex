\chapter{機構の実装}
\label{chp:first}

\section{予備実験}
\label{sec:paragraph}

水に砂糖等の粘度を上げられる物質を添加し造形を行う方式の実証を行った.
初めに,3D プリンターとして自動化させる前に,水の粘度が造形物の造形速度,造形精度,オーバーハングの造形に影響を与えるのか調査を行った. 
造形の仕組みは図のようになっている.実験の装置は, 保温のため一番下に発泡スチロールの容器を用意した.
その上に-196 度の液体窒素を十分に注ぎ,さらにその上からアルミトレーを沈める.
それにより,アルミトレーも液体窒素に近い温度まで冷やされ,そこに注射器を使い水あめと水の中間の粘度の水をたらすことで,冷やされた水が氷に変わる.
冷やされた氷の温度は 0 度よりも低く,その上に水をたらすと氷柱ができるように氷が積層される. 制作した装置は図 2 である。この装置を使い,水の粘度 がどの程度、造形速度と造形精度,オーバーハングの造形に影響するのか調査を行った.
造形物はオーバーハングの調査を行うため図のように中を空洞になるように造形を行った. 実際に制作した装置を使い造形を行ったものが図である.造形時間は約 5 分ほどで完成した.造形精度の問題もあるが通常の3Dプリンターよりもかなり早い結果になった.
大きさは横幅約 3 センチ,高さ約 1.5 センチほどである.使用した水の量は,約 200ml ほどである.初めに想定した形通りに造形ができ,オーバーハングの造形も成功した.また,発見した特徴として,透明度の高い氷を制作することができた.

\section{造形の仕組み}
\label{sec:paragraph}
予備実験では,液体窒素と水に砂糖を混ぜ粘度を上げることにより,ある程度の精度と速度を持つ事が分かった.
ここでは,予備実験を自動化させ,プリンターが氷を積層造形していく仕組みについて解説する.
造形用のペットに熱伝導率の高い金属製のアルミプレートを使用し、液体窒素の保温性を高める為発泡スチロールでできた容器に沈めた.
液体窒素は-196℃であり,アルミプレートもそれに近い温度まで冷やされる.そこに水をたらすことで,水が冷やされ氷が作られる.
氷は,アルミプレートを通して、液体窒素により冷やされ続けるため,氷の温度も0℃以下になる,その上に水をたらすとその水も氷へと状態が変化し氷が積層される.

\section{液体窒素を用いた造形の実装}
\label{sec:paragraph}
ここでは,液体窒素を使用した造形機構について述べる.
機構の全体像は図のようになっている.
ノズルから水を供給するためのシリンジを押し出す機構を実装した.
また,今回の水は粘度を持たせているため,長いチューブを用いてしまうと,抵抗で押し出すのが難しくなる.そのためシリンジからノズルまでの距離ができるだけ短くなる機構を実装した.

\section{プリンターの制御}
\label{sec:paragraph}
氷の3Dプリンターの制御は,Marlin-Ai3Mという3Dプリンターの制御用アプリケーションとMarilnというファームウェアを一部改造し使用している.
改造内容は,モーターの駆動方向の変更,モータードライバーへの対応,3Dプリンターは安全装置として,一定の温度以下で作動しないようになっている.
この安全装置が0℃以下で稼働する氷の3Dプリンターでは必要が無いため,無効にさせた.
この作業を行ったことにより,基本的に一般に販売されている3Dプリンターと同じように制御することができる.

\section{シリンジの機構}
\label{sec:paragraph}
シリンジを押し出して水を供給するために,既存のエクストルーダー用のモーターを利用している.
シリンジを押し出すためにモーターの回転を上下の運動へ置き換えるために、全ねじ棒を使用した.また,そのままモーターを直結してしまうと,力不足になることが想定されてため,ギヤで回転数を調整している.
既存の3Dプリンターでは、フィラメントを押し出すモーターを利用しているため,PCを使い水の押し出し量を自由に調整することが可能であり,安定して造形ができる設定を模索することができる.

\section{ベッドの機構}
\label{sec:paragraph}
液体窒素で氷を造形するベッドとして、大きく2つに分けられる.1つ目が液体窒素用のトレーだ.液体窒素用のトレーでは,下に保温性を高め、液体窒素の持ち時間を長くするために、発泡スチロールを使用した.
また,アルミプレートの下に液体窒素がたまるようにくぼみをつけている.2つ目が造形用のトレーだ.熱伝導率の高い金属のプレートを使用する.今回はアルミ製のプレートを使用した.
また,造形したものを取り出しやすくするために、取り外しが容易な設計を行った.完成したハードは図のようになっている.


