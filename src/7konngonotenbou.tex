\chapter{結果と考察}
\label{chp:first}

今回の実験では,氷をマテリアルとした3Dプリンターの開発と氷のマテリアルとした造形を作成する際,純粋な水ではなく,砂糖を混ぜ粘性を持たせた液体を用いて造形を行った方がより精度の高い造形がでいるのではないかと仮説をたて検証を行った.
実験結果をとして,純粋な水を使用した場合と水と砂糖を1:3で混ぜた液体での場合で比較を行った.
水と砂糖を混ぜた液体での方が造形精度が高く,積層もスムーズに行われた.
よって,氷をマテリアルとした3Dプリンターでの造形を行う際,水と砂糖を混ぜた液体の方が氷の条件を満たしたうえで,適しているということが分かった.

また,氷をマテリアルとした3Dプリンターの開発において,使用したものは既存の3Dプリンターの部品やどこでも手に入れやすいものを多く使用した.
特に,一般で広く使われているスライサーソフトUltimaker Curaを利用して造形できるのは,3Dプリンターを扱えるもの方なら,だれでも操作可能なものになっている.

よって,既存で開発されている氷をマテリアルとした3Dプリンターよりに比べ,
一般の人でも扱いが可能かつ,一般のユーザーが設計したデータにできるだけ近い形に印刷できる氷をマテリアルとした3Dプリンターの開発ができたと考える.

しかし,今回開発した氷をマテリアルとした3Dプリンターにもいくつか課題が残る.
1つ目が予備実験において,手で試した際の精度を自動かさせた際に再現できていないこと.
2つ目一度に造形できる量に制限があること.
この二つを解決できたらと考える.
1つ目の予備実験において,手で試した際の精度を自動かさせた際に再現できていないこととくに,オーバーハングに関しては,いくつか原因が考えられる.
手で試した際に使用した,液体の温度が50℃近くと温かかったものに対して,実験で使用したものは常温だった.
液体の温度がある程度高温の際,既に造形されている氷を一部溶かして積層される.
一度解けてから,再度氷になっているため,層ごとの結びつきが強くなっる.
これと,液体の粘度が加わることにより,予備実験ではオーバーハングが可能であったと考える.
そのため,今後押し出す液体の温度につていの検証が必要であると考える.
また,今回ノズルの大きさ 1.0mm で行ったが,ほかの大きさでどのような違いが現れるのか,最適なノズルのサイズを探す必要もあると考える.

2つ目の一度に造形ができる量が制限があることに関しては,
今回液体を押し出す機構とノズルの距離を粘度を持つ流体が通るため、抵抗が大きくなると予想されたため,シリンダーを採用した.
今後ポンプを利用した,液体の押し出し機構や



今後ためしたこと
┣粘度の最適化
┣ベットの最適化
┣

