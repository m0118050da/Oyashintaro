\chapter{今後の展望}
\label{chp:first}

\section{今後の課題と予定}
\label{sec:paragraph}


問題として,私の現状の研究ではオーバーハングのある造形物は作ることができない.上部の
3 構造物が土台の氷よりも大きい場合,ガス式では上から吹き付けて造形する仕組みの都合で,造
4 形することができない.空冷式は,水を垂らすため地面のない場所には氷を作ることができない.
5 同様に中身が埋まっていない形状では氷を積むことができないため,造形が不可能になっている.
6 オーバーハングの造形をする際は氷同士が接着する性質を生かし,平面のオブジェクトはある程
7 度の精度を持って造形することが可能なため,GCode を分割し造形したパーツを接着させること
8 で,造形の幅が大きく広がる.これには,GCode を読み込んだ際にオーバーハングしている部分
9 を分割し,別の GCode にするシステムが必要になる.あるいは,ラフトを使い氷のサポートを作
10 りながら造形する必要がある.ラフトを使う場合は,塩水を使うか色つきの水で剥がすべき部分
11 の氷をわかるようにしておく必要がある.
12 フロンガスを利用した 3D プリンタは,HFC という代替フロンを利用している.モントリオー
13 ル議定書で,特にオゾン層破壊に影響が強いとされる特定フロンとは違い,オゾンを破壊しにく
14 いフロンになっている.造形時に独特な風味が残るが,フロン自体は人体に影響はないため食し
15 ても問題はない.しかし,味が悪いことと環境のことを考えると二酸化炭素を使うべきであると
16 考える.二酸化炭素は自然界にありふれた物質で,フロンよりもオゾンを破壊しづらく人体に影
17 響もない.炭酸水などに用いられており氷に使用しても問題ないと考える.
18 今後は,オーバーハングの問題を解決するための手法を試す必要がある.