\chapter{ }
\label{chp:first}
本研究のために書き換えた3DプリンターのファームウエアMariln内にある,変更ファイルConfiguration.hのコードである. 



\begin{lstlisting}



  /**
 * Marlin 3D Printer Firmware
 * Copyright (C) 2016 MarlinFirmware [https://github.com/MarlinFirmware/Marlin]
 *
 * Based on Sprinter and grbl.
 * Copyright (C) 2011 Camiel Gubbels / Erik van der Zalm
 *
 * This program is free software: you can redistribute it and/or modify
 * it under the terms of the GNU General Public License as published by
 * the Free Software Foundation, either version 3 of the License, or
 * (at your option) any later version.
 *
 * This program is distributed in the hope that it will be useful,
 * but WITHOUT ANY WARRANTY; without even the implied warranty of
 * MERCHANTABILITY or FITNESS FOR A PARTICULAR PURPOSE.  See the
 * GNU General Public License for more details.
 *
 * You should have received a copy of the GNU General Public License
 * along with this program.  If not, see <http://www.gnu.org/licenses/>.
 *
 */

/**
 * Configuration.h
 *
 * Basic settings such as:
 *
 * - Type of electronics
 * - Type of temperature sensor
 * - Printer geometry
 * - Endstop configuration
 * - LCD controller
 * - Extra features
 *
 * Advanced settings can be found in Configuration_adv.h
 *
 */
#ifndef CONFIGURATION_H
#define CONFIGURATION_H
#define CONFIGURATION_H_VERSION 010109

//===========================================================================
//============================= Getting Started =============================
//===========================================================================

/**
 * Here are some standard links for getting your machine calibrated:
 *
 * http://reprap.org/wiki/Calibration
 * http://youtu.be/wAL9d7FgInk
 * http://calculator.josefprusa.cz
 * http://reprap.org/wiki/Triffid_Hunter%27s_Calibration_Guide
 * http://www.thingiverse.com/thing:5573
 * https://sites.google.com/site/repraplogphase/calibration-of-your-reprap
 * http://www.thingiverse.com/thing:298812
 */

//===========================================================================
//============================= DELTA Printer ===============================
//===========================================================================
// For a Delta printer start with one of the configuration files in the
// example_configurations/delta directory and customize for your machine.
//

//===========================================================================
//============================= SCARA Printer ===============================
//===========================================================================
// For a SCARA printer start with the configuration files in
// example_configurations/SCARA and customize for your machine.
//

//===========================================================================
//============================= HANGPRINTER =================================
//===========================================================================
// For a Hangprinter start with the configuration file in the
// example_configurations/hangprinter directory and customize for your machine.
//

// @section info

// User-specified version info of this build to display in [Pronterface, etc] terminal window during
// startup. Implementation of an idea by Prof Braino to inform user that any changes made to this
// build by the user have been successfully uploaded into firmware.
#define STRING_CONFIG_H_AUTHOR "(davidramiro)" // Who made the changes.
#define SHOW_BOOTSCREEN
#define STRING_SPLASH_LINE1 SHORT_BUILD_VERSION   // will be shown during bootup in line 1
#define STRING_SPLASH_LINE2 CUSTOM_BUILD_VERSION  // will be shown during bootup in line 2

/**
 * *** VENDORS PLEASE READ ***
 *
 * Marlin allows you to add a custom boot image for Graphical LCDs.
 * With this option Marlin will first show your custom screen followed
 * by the standard Marlin logo with version number and web URL.
 *
 * We encourage you to take advantage of this new feature and we also
 * respectfully request that you retain the unmodified Marlin boot screen.
 */

// Enable to show the bitmap in Marlin/_Bootscreen.h on startup.
//#define SHOW_CUSTOM_BOOTSCREEN

// Enable to show the bitmap in Marlin/_Statusscreen.h on the status screen.
//#define CUSTOM_STATUS_SCREEN_IMAGE

// @section machine

/**
 * Select the serial port on the board to use for communication with the host.
 * This allows the connection of wireless adapters (for instance) to non-default port pins.
 * Serial port 0 is always used by the Arduino bootloader regardless of this setting.
 *
 * :[0, 1, 2, 3, 4, 5, 6, 7]
 */
#define SERIAL_PORT 0

/**
 * This setting determines the communication speed of the printer.
 *
 * 250000 works in most cases, but you might try a lower speed if
 * you commonly experience drop-outs during host printing.
 * You may try up to 1000000 to speed up SD file transfer.
 *
 * :[2400, 9600, 19200, 38400, 57600, 115200, 250000, 500000, 1000000]
 */
#define BAUDRATE 250000

// Enable the Bluetooth serial interface on AT90USB devices
//#define BLUETOOTH

// The following define selects which electronics board you have.
// Please choose the name from boards.h that matches your setup
#ifndef MOTHERBOARD
  #define MOTHERBOARD BOARD_TRIGORILLA_14
#endif

// Optional custom name for your RepStrap or other custom machine
// Displayed in the LCD "Ready" message
//#define CUSTOM_MACHINE_NAME "3D Printer"

// Define this to set a unique identifier for this printer, (Used by some programs to differentiate between machines)
// You can use an online service to generate a random UUID. (eg http://www.uuidgenerator.net/version4)
//#define MACHINE_UUID "00000000-0000-0000-0000-000000000000"

// @section extruder

// This defines the number of extruders
// :[1, 2, 3, 4, 5]
#define EXTRUDERS 1

// Generally expected filament diameter (1.75, 2.85, 3.0, ...). Used for Volumetric, Filament Width Sensor, etc.
#define DEFAULT_NOMINAL_FILAMENT_DIA 1.75

// For Cyclops or any "multi-extruder" that shares a single nozzle.
//#define SINGLENOZZLE

/**
 * Prusa MK2 Single Nozzle Multi-Material Multiplexer, and variants.
 *
 * This device allows one stepper driver on a control board to drive
 * two to eight stepper motors, one at a time, in a manner suitable
 * for extruders.
 *
 * This option only allows the multiplexer to switch on tool-change.
 * Additional options to configure custom E moves are pending.
 */
//#define MK2_MULTIPLEXER
#if ENABLED(MK2_MULTIPLEXER)
  // Override the default DIO selector pins here, if needed.
  // Some pins files may provide defaults for these pins.
  //#define E_MUX0_PIN 40  // Always Required
  //#define E_MUX1_PIN 42  // Needed for 3 to 8 steppers
  //#define E_MUX2_PIN 44  // Needed for 5 to 8 steppers
#endif

// A dual extruder that uses a single stepper motor
//#define SWITCHING_EXTRUDER
#if ENABLED(SWITCHING_EXTRUDER)
  #define SWITCHING_EXTRUDER_SERVO_NR 0
  #define SWITCHING_EXTRUDER_SERVO_ANGLES { 0, 90 } // Angles for E0, E1[, E2, E3]
  #if EXTRUDERS > 3
    #define SWITCHING_EXTRUDER_E23_SERVO_NR 1
  #endif
#endif

// A dual-nozzle that uses a servomotor to raise/lower one of the nozzles
//#define SWITCHING_NOZZLE
#if ENABLED(SWITCHING_NOZZLE)
  #define SWITCHING_NOZZLE_SERVO_NR 0
  #define SWITCHING_NOZZLE_SERVO_ANGLES { 0, 90 }   // Angles for E0, E1
  //#define HOTEND_OFFSET_Z { 0.0, 0.0 }
#endif

/**
 * Two separate X-carriages with extruders that connect to a moving part
 * via a magnetic docking mechanism. Requires SOL1_PIN and SOL2_PIN.
 */
//#define PARKING_EXTRUDER
#if ENABLED(PARKING_EXTRUDER)
  #define PARKING_EXTRUDER_SOLENOIDS_INVERT           // If enabled, the solenoid is NOT magnetized with applied voltage
  #define PARKING_EXTRUDER_SOLENOIDS_PINS_ACTIVE LOW  // LOW or HIGH pin signal energizes the coil
  #define PARKING_EXTRUDER_SOLENOIDS_DELAY 250        // Delay (ms) for magnetic field. No delay if 0 or not defined.
  #define PARKING_EXTRUDER_PARKING_X { -78, 184 }     // X positions for parking the extruders
  #define PARKING_EXTRUDER_GRAB_DISTANCE 1            // mm to move beyond the parking point to grab the extruder
  #define PARKING_EXTRUDER_SECURITY_RAISE 5           // Z-raise before parking
  #define HOTEND_OFFSET_Z { 0.0, 1.3 }                // Z-offsets of the two hotends. The first must be 0.
#endif

/**
 * "Mixing Extruder"
 *   - Adds G-codes M163 and M164 to set and "commit" the current mix factors.
 *   - Extends the stepping routines to move multiple steppers in proportion to the mix.
 *   - Optional support for Repetier Firmware's 'M164 S<index>' supporting virtual tools.
 *   - This implementation supports up to two mixing extruders.
 *   - Enable DIRECT_MIXING_IN_G1 for M165 and mixing in G1 (from Pia Taubert's reference implementation).
 */
//#define MIXING_EXTRUDER
#if ENABLED(MIXING_EXTRUDER)
  #define MIXING_STEPPERS 2        // Number of steppers in your mixing extruder
  #define MIXING_VIRTUAL_TOOLS 16  // Use the Virtual Tool method with M163 and M164
  //#define DIRECT_MIXING_IN_G1    // Allow ABCDHI mix factors in G1 movement commands
#endif

// Offset of the extruders (uncomment if using more than one and relying on firmware to position when changing).
// The offset has to be X=0, Y=0 for the extruder 0 hotend (default extruder).
// For the other hotends it is their distance from the extruder 0 hotend.
//#define HOTEND_OFFSET_X {0.0, 20.00} // (in mm) for each extruder, offset of the hotend on the X axis
//#define HOTEND_OFFSET_Y {0.0, 5.00}  // (in mm) for each extruder, offset of the hotend on the Y axis

// @section machine

/**
 * Select your power supply here. Use 0 if you haven't connected the PS_ON_PIN
 *
 * 0 = No Power Switch
 * 1 = ATX
 * 2 = X-Box 360 203Watts (the blue wire connected to PS_ON and the red wire to VCC)
 *
 * :{ 0:'No power switch', 1:'ATX', 2:'X-Box 360' }
 */
#define POWER_SUPPLY 0

#if POWER_SUPPLY > 0
  // Enable this option to leave the PSU off at startup.
  // Power to steppers and heaters will need to be turned on with M80.
  //#define PS_DEFAULT_OFF

  //#define AUTO_POWER_CONTROL        // Enable automatic control of the PS_ON pin
  #if ENABLED(AUTO_POWER_CONTROL)
    #define AUTO_POWER_FANS           // Turn on PSU if fans need power
    #define AUTO_POWER_E_FANS
    #define AUTO_POWER_CONTROLLERFAN
    #define POWER_TIMEOUT 30
  #endif

#endif

// @section temperature

//===========================================================================
//============================= Thermal Settings ============================
//===========================================================================

/**
 * --NORMAL IS 4.7kohm PULLUP!-- 1kohm pullup can be used on hotend sensor, using correct resistor and table
 *
 * Temperature sensors available:
 *
 *    -4 : thermocouple with AD8495
 *    -3 : thermocouple with MAX31855 (only for sensor 0)
 *    -2 : thermocouple with MAX6675 (only for sensor 0)
 *    -1 : thermocouple with AD595
 *     0 : not used
 *     1 : 100k thermistor - best choice for EPCOS 100k (4.7k pullup)
 *     2 : 200k thermistor - ATC Semitec 204GT-2 (4.7k pullup)
 *     3 : Mendel-parts thermistor (4.7k pullup)
 *     4 : 10k thermistor !! do not use it for a hotend. It gives bad resolution at high temp. !!
 *     5 : 100K thermistor - ATC Semitec 104GT-2/104NT-4-R025H42G (Used in ParCan & J-Head) (4.7k pullup)
 *   501 : 100K Zonestar (Tronxy X3A) Thermistor
 *     6 : 100k EPCOS - Not as accurate as table 1 (created using a fluke thermocouple) (4.7k pullup)
 *     7 : 100k Honeywell thermistor 135-104LAG-J01 (4.7k pullup)
 *    71 : 100k Honeywell thermistor 135-104LAF-J01 (4.7k pullup)
 *     8 : 100k 0603 SMD Vishay NTCS0603E3104FXT (4.7k pullup)
 *     9 : 100k GE Sensing AL03006-58.2K-97-G1 (4.7k pullup)
 *    10 : 100k RS thermistor 198-961 (4.7k pullup)
 *    11 : 100k beta 3950 1% thermistor (4.7k pullup)
 *    12 : 100k 0603 SMD Vishay NTCS0603E3104FXT (4.7k pullup) (calibrated for Makibox hot bed)
 *    13 : 100k Hisens 3950  1% up to 300°C for hotend "Simple ONE " & "Hotend "All In ONE"
 *    15 : 100k thermistor calibration for JGAurora A5 hotend
 *    20 : the PT100 circuit found in the Ultimainboard V2.x
 *    60 : 100k Maker's Tool Works Kapton Bed Thermistor beta=3950
 *    66 : 4.7M High Temperature thermistor from Dyze Design
 *    70 : the 100K thermistor found in the bq Hephestos 2
 *    75 : 100k Generic Silicon Heat Pad with NTC 100K MGB18-104F39050L32 thermistor
 *
 *       1k ohm pullup tables - This is atypical, and requires changing out the 4.7k pullup for 1k.
 *                              (but gives greater accuracy and more stable PID)
 *    51 : 100k thermistor - EPCOS (1k pullup)
 *    52 : 200k thermistor - ATC Semitec 204GT-2 (1k pullup)
 *    55 : 100k thermistor - ATC Semitec 104GT-2 (Used in ParCan & J-Head) (1k pullup)
 *
 *  1047 : Pt1000 with 4k7 pullup
 *  1010 : Pt1000 with 1k pullup (non standard)
 *   147 : Pt100 with 4k7 pullup
 *   110 : Pt100 with 1k pullup (non standard)
 *
 *         Use these for Testing or Development purposes. NEVER for production machine.
 *   998 : Dummy Table that ALWAYS reads 25°C or the temperature defined below.
 *   999 : Dummy Table that ALWAYS reads 100°C or the temperature defined below.
 *
 * :{ '0': "Not used", '1':"100k / 4.7k - EPCOS", '2':"200k / 4.7k - ATC Semitec 204GT-2", '3':"Mendel-parts / 4.7k", '4':"10k !! do not use for a hotend. Bad resolution at high temp. !!", '5':"100K / 4.7k - ATC Semitec 104GT-2 (Used in ParCan & J-Head)", '501':"100K Zonestar (Tronxy X3A)", '6':"100k / 4.7k EPCOS - Not as accurate as Table 1", '7':"100k / 4.7k Honeywell 135-104LAG-J01", '8':"100k / 4.7k 0603 SMD Vishay NTCS0603E3104FXT", '9':"100k / 4.7k GE Sensing AL03006-58.2K-97-G1", '10':"100k / 4.7k RS 198-961", '11':"100k / 4.7k beta 3950 1%", '12':"100k / 4.7k 0603 SMD Vishay NTCS0603E3104FXT (calibrated for Makibox hot bed)", '13':"100k Hisens 3950  1% up to 300°C for hotend 'Simple ONE ' & hotend 'All In ONE'", '20':"PT100 (Ultimainboard V2.x)", '51':"100k / 1k - EPCOS", '52':"200k / 1k - ATC Semitec 204GT-2", '55':"100k / 1k - ATC Semitec 104GT-2 (Used in ParCan & J-Head)", '60':"100k Maker's Tool Works Kapton Bed Thermistor beta=3950", '66':"Dyze Design 4.7M High Temperature thermistor", '70':"the 100K thermistor found in the bq Hephestos 2", '71':"100k / 4.7k Honeywell 135-104LAF-J01", '147':"Pt100 / 4.7k", '1047':"Pt1000 / 4.7k", '110':"Pt100 / 1k (non-standard)", '1010':"Pt1000 / 1k (non standard)", '-4':"Thermocouple + AD8495", '-3':"Thermocouple + MAX31855 (only for sensor 0)", '-2':"Thermocouple + MAX6675 (only for sensor 0)", '-1':"Thermocouple + AD595",'998':"Dummy 1", '999':"Dummy 2" }
 */
#define TEMP_SENSOR_0 5
#define TEMP_SENSOR_1 0
#define TEMP_SENSOR_2 0
#define TEMP_SENSOR_3 0
#define TEMP_SENSOR_4 0
#define TEMP_SENSOR_BED 1
#define TEMP_SENSOR_CHAMBER 0

// Dummy thermistor constant temperature readings, for use with 998 and 999
#define DUMMY_THERMISTOR_998_VALUE 25
#define DUMMY_THERMISTOR_999_VALUE 100

// Use temp sensor 1 as a redundant sensor with sensor 0. If the readings
// from the two sensors differ too much the print will be aborted.
//#define TEMP_SENSOR_1_AS_REDUNDANT
#define MAX_REDUNDANT_TEMP_SENSOR_DIFF 10

// Extruder temperature must be close to target for this long before M109 returns success
#define TEMP_RESIDENCY_TIME 10  // (seconds)
#define TEMP_HYSTERESIS 3       // (degC) range of +/- temperatures considered "close" to the target one
#define TEMP_WINDOW     1       // (degC) Window around target to start the residency timer x degC early.

// Bed temperature must be close to target for this long before M190 returns success
#define TEMP_BED_RESIDENCY_TIME 10  // (seconds)
#define TEMP_BED_HYSTERESIS 3       // (degC) range of +/- temperatures considered "close" to the target one
#define TEMP_BED_WINDOW     1       // (degC) Window around target to start the residency timer x degC early.

// The minimal temperature defines the temperature below which the heater will not be enabled It is used
// to check that the wiring to the thermistor is not broken.
// Otherwise this would lead to the heater being powered on all the time.
#define HEATER_0_MINTEMP 5
#define HEATER_1_MINTEMP 5
#define HEATER_2_MINTEMP 5
#define HEATER_3_MINTEMP 5
#define HEATER_4_MINTEMP 5
#define BED_MINTEMP 5

// When temperature exceeds max temp, your heater will be switched off.
// This feature exists to protect your hotend from overheating accidentally, but *NOT* from thermistor short/failure!
// You should use MINTEMP for thermistor short/failure protection.
#define HEATER_0_MAXTEMP 285
#define HEATER_1_MAXTEMP 275
#define HEATER_2_MAXTEMP 275
#define HEATER_3_MAXTEMP 275
#define HEATER_4_MAXTEMP 275
#define BED_MAXTEMP 135

//===========================================================================
//============================= PID Settings ================================
//===========================================================================
// PID Tuning Guide here: http://reprap.org/wiki/PID_Tuning

// Comment the following line to disable PID and enable bang-bang.
#define PIDTEMP
#define BANG_MAX 255     // Limits current to nozzle while in bang-bang mode; 255=full current
#define PID_MAX BANG_MAX // Limits current to nozzle while PID is active (see PID_FUNCTIONAL_RANGE below); 255=full current
#define PID_K1 0.95      // Smoothing factor within any PID loop
#if ENABLED(PIDTEMP)
  //#define PID_AUTOTUNE_MENU // Add PID Autotune to the LCD "Temperature" menu to run M303 and apply the result.
  //#define PID_DEBUG // Sends debug data to the serial port.
  //#define PID_OPENLOOP 1 // Puts PID in open loop. M104/M140 sets the output power from 0 to PID_MAX
  //#define SLOW_PWM_HEATERS // PWM with very low frequency (roughly 0.125Hz=8s) and minimum state time of approximately 1s useful for heaters driven by a relay
  //#define PID_PARAMS_PER_HOTEND // Uses separate PID parameters for each extruder (useful for mismatched extruders)
                                  // Set/get with gcode: M301 E[extruder number, 0-2]
  #define PID_FUNCTIONAL_RANGE 10 // If the temperature difference between the target temperature and the actual temperature
                                  // is more than PID_FUNCTIONAL_RANGE then the PID will be shut off and the heater will be set to min/max.

  // If you are using a pre-configured hotend then you can use one of the value sets by uncommenting it

  // i3 Mega stock v5 hotend, 40W heater cartridge (3.6Ω @ 22°C)
  #define  DEFAULT_Kp 15.94
  #define  DEFAULT_Ki 1.17
  #define  DEFAULT_Kd 54.19

  // Ultimaker
  //#define  DEFAULT_Kp 22.2
  //#define  DEFAULT_Ki 1.08
  //#define  DEFAULT_Kd 114

  // MakerGear
  //#define DEFAULT_Kp 7.0
  //#define DEFAULT_Ki 0.1
  //#define DEFAULT_Kd 12

  // Mendel Parts V9 on 12V
  //#define DEFAULT_Kp 63.0
  //#define DEFAULT_Ki 2.25
  //#define DEFAULT_Kd 440

#endif // PIDTEMP

//===========================================================================
//============================= PID > Bed Temperature Control ===============
//===========================================================================

/**
 * PID Bed Heating
 *
 * If this option is enabled set PID constants below.
 * If this option is disabled, bang-bang will be used and BED_LIMIT_SWITCHING will enable hysteresis.
 *
 * The PID frequency will be the same as the extruder PWM.
 * If PID_dT is the default, and correct for the hardware/configuration, that means 7.689Hz,
 * which is fine for driving a square wave into a resistive load and does not significantly
 * impact FET heating. This also works fine on a Fotek SSR-10DA Solid State Relay into a 250W
 * heater. If your configuration is significantly different than this and you don't understand
 * the issues involved, don't use bed PID until someone else verifies that your hardware works.
 */
#define PIDTEMPBED

#define MAX_CYCLE_TIME_PID_AUTOTUNE 40L

//#define BED_LIMIT_SWITCHING

/**
 * Max Bed Power
 * Applies to all forms of bed control (PID, bang-bang, and bang-bang with hysteresis).
 * When set to any value below 255, enables a form of PWM to the bed that acts like a divider
 * so don't use it unless you are OK with PWM on your bed. (See the comment on enabling PIDTEMPBED)
 */
#define MAX_BED_POWER 255 // limits duty cycle to bed; 255=full current

#if ENABLED(PIDTEMPBED)

  //#define PID_BED_DEBUG // Sends debug data to the serial port.

  //Anycubic i3 Mega Ultrabase (0.9Ω @ 22°C)
  #define DEFAULT_bedKp 251.78
  #define DEFAULT_bedKi 49.57
  #define DEFAULT_bedKd 319.73

  //120V 250W silicone heater into 4mm borosilicate (MendelMax 1.5+)
  //from pidautotune
  //#define DEFAULT_bedKp 97.1
  //#define DEFAULT_bedKi 1.41
  //#define DEFAULT_bedKd 1675.16

  // FIND YOUR OWN: "M303 E-1 C8 S90" to run autotune on the bed at 90 degreesC for 8 cycles.
#endif // PIDTEMPBED

// @section extruder

/**
 * Prevent extrusion if the temperature is below EXTRUDE_MINTEMP.
 * Add M302 to set the minimum extrusion temperature and/or turn
 * cold extrusion prevention on and off.
 *
 * *** IT IS HIGHLY RECOMMENDED TO LEAVE THIS OPTION ENABLED! ***
 */
#define PREVENT_COLD_EXTRUSION
//変えた
#define EXTRUDE_MINTEMP -5

/**
 * Prevent a single extrusion longer than EXTRUDE_MAXLENGTH.
 * Note: For Bowden Extruders make this large enough to allow load/unload.
 */
#define PREVENT_LENGTHY_EXTRUDE
#define EXTRUDE_MAXLENGTH 600

//===========================================================================
//======================== Thermal Runaway Protection =======================
//===========================================================================

/**
 * Thermal Protection provides additional protection to your printer from damage
 * and fire. Marlin always includes safe min and max temperature ranges which
 * protect against a broken or disconnected thermistor wire.
 *
 * The issue: If a thermistor falls out, it will report the much lower
 * temperature of the air in the room, and the the firmware will keep
 * the heater on.
 *
 * If you get "Thermal Runaway" or "Heating failed" errors the
 * details can be tuned in Configuration_adv.h
 */

#define THERMAL_PROTECTION_HOTENDS // Enable thermal protection for all extruders
#define THERMAL_PROTECTION_BED     // Enable thermal protection for the heated bed

//===========================================================================
//============================= Mechanical Settings =========================
//===========================================================================

// @section machine

// Uncomment one of these options to enable CoreXY, CoreXZ, or CoreYZ kinematics
// either in the usual order or reversed
//#define COREXY
//#define COREXZ
//#define COREYZ
//#define COREYX
//#define COREZX
//#define COREZY

//===========================================================================
//============================== Endstop Settings ===========================
//===========================================================================

// @section homing

// Specify here all the endstop connectors that are connected to any endstop or probe.
// Almost all printers will be using one per axis. Probes will use one or more of the
// extra connectors. Leave undefined any used for non-endstop and non-probe purposes.
#define USE_XMIN_PLUG
#define USE_YMIN_PLUG
#define USE_ZMIN_PLUG
#define USE_XMAX_PLUG
//#define USE_YMAX_PLUG
#define USE_ZMAX_PLUG

// Enable pullup for all endstops to prevent a floating state
#define ENDSTOPPULLUPS
#if DISABLED(ENDSTOPPULLUPS)
  // Disable ENDSTOPPULLUPS to set pullups individually
  //#define ENDSTOPPULLUP_XMAX
  //#define ENDSTOPPULLUP_YMAX
  //#define ENDSTOPPULLUP_ZMAX
  //#define ENDSTOPPULLUP_XMIN
  //#define ENDSTOPPULLUP_YMIN
  //#define ENDSTOPPULLUP_ZMIN
  //#define ENDSTOPPULLUP_ZMIN_PROBE
#endif

// Mechanical endstop with COM to ground and NC to Signal uses "false" here (most common setup).
#define X_MIN_ENDSTOP_INVERTING true // set to true to invert the logic of the endstop.
#define Y_MIN_ENDSTOP_INVERTING true // set to true to invert the logic of the endstop.
#define Z_MIN_ENDSTOP_INVERTING true // set to true to invert the logic of the endstop.
#define X_MAX_ENDSTOP_INVERTING true // set to true to invert the logic of the endstop.
#define Y_MAX_ENDSTOP_INVERTING true // set to true to invert the logic of the endstop.
#define Z_MAX_ENDSTOP_INVERTING true // set to true to invert the logic of the endstop.
#define Z_MIN_PROBE_ENDSTOP_INVERTING true // set to true to invert the logic of the probe.

/**
 * Stepper Drivers
 *
 * These settings allow Marlin to tune stepper driver timing and enable advanced options for
 * stepper drivers that support them. You may also override timing options in Configuration_adv.h.
 *
 * A4988 is assumed for unspecified drivers.
 *
 * Options: A4988, DRV8825, LV8729, L6470, TB6560, TB6600, TMC2100,
 *          TMC2130, TMC2130_STANDALONE, TMC2208, TMC2208_STANDALONE,
 *          TMC26X,  TMC26X_STANDALONE,  TMC2660, TMC2660_STANDALONE,
 *          TMC5130, TMC5130_STANDALONE
 * :['A4988', 'DRV8825', 'LV8729', 'L6470', 'TB6560', 'TB6600', 'TMC2100', 'TMC2130', 'TMC2130_STANDALONE', 'TMC2208', 'TMC2208_STANDALONE', 'TMC26X', 'TMC26X_STANDALONE', 'TMC2660', 'TMC2660_STANDALONE', 'TMC5130', 'TMC5130_STANDALONE']
 */
#define X_DRIVER_TYPE  A4988 // comment out for stock drivers
#define Y_DRIVER_TYPE  A4998 // comment out for stock drivers
#define Z_DRIVER_TYPE  A4998 // comment out for stock drivers
#define X2_DRIVER_TYPE A4998
#define Y2_DRIVER_TYPE A4998
#define Z2_DRIVER_TYPE A4998 // comment out for stock drivers
#define E0_DRIVER_TYPE A4998 // comment out for stock drivers
#define E1_DRIVER_TYPE A4998 // comment out for stock drivers
#define E2_DRIVER_TYPE A4998
#define E3_DRIVER_TYPE A4998
#define E4_DRIVER_TYPE A4998

// Enable this feature if all enabled endstop pins are interrupt-capable.
// This will remove the need to poll the interrupt pins, saving many CPU cycles.
//#define ENDSTOP_INTERRUPTS_FEATURE

/**
 * Endstop Noise Filter
 *
 * Enable this option if endstops falsely trigger due to noise.
 * NOTE: Enabling this feature means adds an error of +/-0.2mm, so homing
 * will end up at a slightly different position on each G28. This will also
 * reduce accuracy of some bed probes.
 * For mechanical switches, the better approach to reduce noise is to install
 * a 100 nanofarads ceramic capacitor in parallel with the switch, making it
 * essentially noise-proof without sacrificing accuracy.
 * This option also increases MCU load when endstops or the probe are enabled.
 * So this is not recommended. USE AT YOUR OWN RISK.
 * (This feature is not required for common micro-switches mounted on PCBs
 * based on the Makerbot design, since they already include the 100nF capacitor.)
 */
#define ENDSTOP_NOISE_FILTER

//=============================================================================
//============================== Movement Settings ============================
//=============================================================================
// @section motion

/**
 * Default Settings
 *
 * These settings can be reset by M502
 *
 * Note that if EEPROM is enabled, saved values will override these.
 */

/**
 * With this option each E stepper can have its own factors for the
 * following movement settings. If fewer factors are given than the
 * total number of extruders, the last value applies to the rest.
 */
//#define DISTINCT_E_FACTORS

/**
 * Default Axis Steps Per Unit (steps/mm)
 * Override with M92
 *                                      X, Y, Z, E0 [, E1[, E2[, E3[, E4]]]]
 */

// ここちょい変えた。
#define DEFAULT_AXIS_STEPS_PER_UNIT   { 80, 80, 400, 500 }

/**
 * Default Max Feed Rate (mm/s)
 * Override with M203
 *                                      X, Y, Z, E0 [, E1[, E2[, E3[, E4]]]]
 */
#define DEFAULT_MAX_FEEDRATE          { 500, 500, 6, 60 }

/**
 * Default Max Acceleration (change/s) change = mm/s
 * (Maximum start speed for accelerated moves)
 * Override with M201
 *                                      X, Y, Z, E0 [, E1[, E2[, E3[, E4]]]]
 */
#define DEFAULT_MAX_ACCELERATION      { 3000, 2000,  60, 10000 }

/**
 * Default Acceleration (change/s) change = mm/s
 * Override with M204
 *
 *   M204 P    Acceleration
 *   M204 R    Retract Acceleration
 *   M204 T    Travel Acceleration
 */
#define DEFAULT_ACCELERATION          1500    // X, Y, Z and E acceleration for printing moves
#define DEFAULT_RETRACT_ACCELERATION  3000    // E acceleration for retracts
#define DEFAULT_TRAVEL_ACCELERATION   3000    // X, Y, Z acceleration for travel (non printing) moves

/**
 * Default Jerk (mm/s)
 * Override with M205 X Y Z E
 *
 * "Jerk" specifies the minimum speed change that requires acceleration.
 * When changing speed and direction, if the difference is less than the
 * value set here, it may happen instantaneously.
 */
#define DEFAULT_XJERK                 10.0
#define DEFAULT_YJERK                 10.0
#define DEFAULT_ZJERK                  0.4
#define DEFAULT_EJERK                  5.0

/**
 * S-Curve Acceleration
 *
 * This option eliminates vibration during printing by fitting a Bezier
 * curve to move acceleration, producing much smoother direction changes.
 *
 * See https://github.com/synthetos/TinyG/wiki/Jerk-Controlled-Motion-Explained
 */
#define S_CURVE_ACCELERATION

//===========================================================================
//============================= Z Probe Options =============================
//===========================================================================
// @section probes

//
// See http://marlinfw.org/docs/configuration/probes.html
//

/**
 * Z_MIN_PROBE_USES_Z_MIN_ENDSTOP_PIN
 *
 * Enable this option for a probe connected to the Z Min endstop pin.
 */
//#define Z_MIN_PROBE_USES_Z_MIN_ENDSTOP_PIN

/**
 * Z_MIN_PROBE_ENDSTOP
 *
 * Enable this option for a probe connected to any pin except Z-Min.
 * (By default Marlin assumes the Z-Max endstop pin.)
 * To use a custom Z Probe pin, set Z_MIN_PROBE_PIN below.
 *
 *  - The simplest option is to use a free endstop connector.
 *  - Use 5V for powered (usually inductive) sensors.
 *
 *  - RAMPS 1.3/1.4 boards may use the 5V, GND, and Aux4->D32 pin:
 *    - For simple switches connect...
 *      - normally-closed switches to GND and D32.
 *      - normally-open switches to 5V and D32.
 *
 * WARNING: Setting the wrong pin may have unexpected and potentially
 * disastrous consequences. Use with caution and do your homework.
 *
 */
#define Z_MIN_PROBE_ENDSTOP

/**
 * Probe Type
 *
 * Allen Key Probes, Servo Probes, Z-Sled Probes, FIX_MOUNTED_PROBE, etc.
 * Activate one of these to use Auto Bed Leveling below.
 */

/**
 * The "Manual Probe" provides a means to do "Auto" Bed Leveling without a probe.
 * Use G29 repeatedly, adjusting the Z height at each point with movement commands
 * or (with LCD_BED_LEVELING) the LCD controller.
 */
#define PROBE_MANUALLY
//#define MANUAL_PROBE_START_Z 0.2

/**
 * A Fix-Mounted Probe either doesn't deploy or needs manual deployment.
 *   (e.g., an inductive probe or a nozzle-based probe-switch.)
 */
//#define FIX_MOUNTED_PROBE

/**
 * Z Servo Probe, such as an endstop switch on a rotating arm.
 */
//#define Z_PROBE_SERVO_NR 0   // Defaults to SERVO 0 connector.
//#define Z_SERVO_ANGLES {70,0}  // Z Servo Deploy and Stow angles

/**
 * The BLTouch probe uses a Hall effect sensor and emulates a servo.
 */
//#define BLTOUCH

/**
 * Enable one or more of the following if probing seems unreliable.
 * Heaters and/or fans can be disabled during probing to minimize electrical
 * noise. A delay can also be added to allow noise and vibration to settle.
 * These options are most useful for the BLTouch probe, but may also improve
 * readings with inductive probes and piezo sensors.
 */
//#define PROBING_HEATERS_OFF       // Turn heaters off when probing
#if ENABLED(PROBING_HEATERS_OFF)
  //#define WAIT_FOR_BED_HEATER     // Wait for bed to heat back up between probes (to improve accuracy)
#endif
//#define PROBING_FANS_OFF          // Turn fans off when probing
//#define DELAY_BEFORE_PROBING 200  // (ms) To prevent vibrations from triggering piezo sensors

// A probe that is deployed and stowed with a solenoid pin (SOL1_PIN)
//#define SOLENOID_PROBE

// A sled-mounted probe like those designed by Charles Bell.
//#define Z_PROBE_SLED
//#define SLED_DOCKING_OFFSET 5  // The extra distance the X axis must travel to pickup the sled. 0 should be fine but you can push it further if you'd like.

//
// For Z_PROBE_ALLEN_KEY see the Delta example configurations.
//

/**
 *   Z Probe to nozzle (X,Y) offset, relative to (0, 0).
 *   X and Y offsets must be integers.
 *
 *   In the following example the X and Y offsets are both positive:
 *   #define X_PROBE_OFFSET_FROM_EXTRUDER 10
 *   #define Y_PROBE_OFFSET_FROM_EXTRUDER 10
 *
 *      +-- BACK ---+
 *      |           |
 *    L |    (+) P  | R <-- probe (20,20)
 *    E |           | I
 *    F | (-) N (+) | G <-- nozzle (10,10)
 *    T |           | H
 *      |    (-)    | T
 *      |           |
 *      O-- FRONT --+
 *    (0,0)
 */
#define X_PROBE_OFFSET_FROM_EXTRUDER 10  // X offset: -left  +right  [of the nozzle]
#define Y_PROBE_OFFSET_FROM_EXTRUDER 10  // Y offset: -front +behind [the nozzle]
#define Z_PROBE_OFFSET_FROM_EXTRUDER 0   // Z offset: -below +above  [the nozzle]

// Certain types of probes need to stay away from edges
#define MIN_PROBE_EDGE 10

// X and Y axis travel speed (mm/m) between probes
#define XY_PROBE_SPEED 8000

// Feedrate (mm/m) for the first approach when double-probing (MULTIPLE_PROBING == 2)
#define Z_PROBE_SPEED_FAST HOMING_FEEDRATE_Z

// Feedrate (mm/m) for the "accurate" probe of each point
#define Z_PROBE_SPEED_SLOW (Z_PROBE_SPEED_FAST / 2)

// The number of probes to perform at each point.
//   Set to 2 for a fast/slow probe, using the second probe result.
//   Set to 3 or more for slow probes, averaging the results.
//#define MULTIPLE_PROBING 2

/**
 * Z probes require clearance when deploying, stowing, and moving between
 * probe points to avoid hitting the bed and other hardware.
 * Servo-mounted probes require extra space for the arm to rotate.
 * Inductive probes need space to keep from triggering early.
 *
 * Use these settings to specify the distance (mm) to raise the probe (or
 * lower the bed). The values set here apply over and above any (negative)
 * probe Z Offset set with Z_PROBE_OFFSET_FROM_EXTRUDER, M851, or the LCD.
 * Only integer values >= 1 are valid here.
 *
 * Example: `M851 Z-5` with a CLEARANCE of 4  =>  9mm from bed to nozzle.
 *     But: `M851 Z+1` with a CLEARANCE of 2  =>  2mm from bed to nozzle.
 */
#define Z_CLEARANCE_DEPLOY_PROBE   10 // Z Clearance for Deploy/Stow
#define Z_CLEARANCE_BETWEEN_PROBES  5 // Z Clearance between probe points
#define Z_CLEARANCE_MULTI_PROBE     5 // Z Clearance between multiple probes
//#define Z_AFTER_PROBING           5 // Z position after probing is done

#define Z_PROBE_LOW_POINT          -2 // Farthest distance below the trigger-point to go before stopping

// For M851 give a range for adjusting the Z probe offset
#define Z_PROBE_OFFSET_RANGE_MIN -20
#define Z_PROBE_OFFSET_RANGE_MAX 20

// Enable the M48 repeatability test to test probe accuracy
//#define Z_MIN_PROBE_REPEATABILITY_TEST

// For Inverting Stepper Enable Pins (Active Low) use 0, Non Inverting (Active High) use 1
// :{ 0:'Low', 1:'High' }
#define X_ENABLE_ON 0
#define Y_ENABLE_ON 0
#define Z_ENABLE_ON 0
#define E_ENABLE_ON 0 // For all extruders

// Disables axis stepper immediately when it's not being used.
// WARNING: When motors turn off there is a chance of losing position accuracy!
#define DISABLE_X false
#define DISABLE_Y false
#define DISABLE_Z false
// Warn on display about possibly reduced accuracy
//#define DISABLE_REDUCED_ACCURACY_WARNING

// @section extruder

#define DISABLE_E false // For all extruders
#define DISABLE_INACTIVE_EXTRUDER true // Keep only the active extruder enabled.

// @section machine

// Invert the stepper direction. Change (or reverse the motor connector) if an axis goes the wrong way.
#define INVERT_X_DIR true // set to true for stock drivers or TMC2208 with reversed connectors
#define INVERT_Y_DIR false // set to false for stock drivers or TMC2208 with reversed connectors
#define INVERT_Z_DIR false // set to false for stock drivers or TMC2208 with reversed connectors

// @section extruder

// For direct drive extruder v9 set to true, for geared extruder set to false.
#define INVERT_E0_DIR false // set to false for stock drivers or TMC2208 with reversed connectors
#define INVERT_E1_DIR false // set to false for stock drivers or TMC2208 with reversed connectors
#define INVERT_E2_DIR false
#define INVERT_E3_DIR false
#define INVERT_E4_DIR false

// @section homing

//#define NO_MOTION_BEFORE_HOMING  // Inhibit movement until all axes have been homed

//#define UNKNOWN_Z_NO_RAISE // Don't raise Z (lower the bed) if Z is "unknown." For beds that fall when Z is powered off.

//#define Z_HOMING_HEIGHT 4  // (in mm) Minimal z height before homing (G28) for Z clearance above the bed, clamps, ...
                             // Be sure you have this distance over your Z_MAX_POS in case.

// Direction of endstops when homing; 1=MAX, -1=MIN
// :[-1,1]
#define X_HOME_DIR -1
#define Y_HOME_DIR -1
#define Z_HOME_DIR -1

// @section machine

// The size of the print bed
#define X_BED_SIZE 215
#define Y_BED_SIZE 215

// Travel limits (mm) after homing, corresponding to endstop positions.
#define X_MIN_POS -5
#define Y_MIN_POS 0
#define Z_MIN_POS 0
#define X_MAX_POS X_BED_SIZE
#define Y_MAX_POS Y_BED_SIZE
#define Z_MAX_POS 205

/**
 * Software Endstops
 *
 * - Prevent moves outside the set machine bounds.
 * - Individual axes can be disabled, if desired.
 * - X and Y only apply to Cartesian robots.
 * - Use 'M211' to set software endstops on/off or report current state
 */

// Min software endstops constrain movement within minimum coordinate bounds
#define MIN_SOFTWARE_ENDSTOPS
#if ENABLED(MIN_SOFTWARE_ENDSTOPS)
  #define MIN_SOFTWARE_ENDSTOP_X
  #define MIN_SOFTWARE_ENDSTOP_Y
  #define MIN_SOFTWARE_ENDSTOP_Z
#endif

// Max software endstops constrain movement within maximum coordinate bounds
#define MAX_SOFTWARE_ENDSTOPS
#if ENABLED(MAX_SOFTWARE_ENDSTOPS)
  #define MAX_SOFTWARE_ENDSTOP_X
  #define MAX_SOFTWARE_ENDSTOP_Y
  #define MAX_SOFTWARE_ENDSTOP_Z
#endif

#if ENABLED(MIN_SOFTWARE_ENDSTOPS) || ENABLED(MAX_SOFTWARE_ENDSTOPS)
  //#define SOFT_ENDSTOPS_MENU_ITEM  // Enable/Disable software endstops from the LCD
#endif

/**
 * Filament Runout Sensors
 * Mechanical or opto endstops are used to check for the presence of filament.
 *
 * RAMPS-based boards use SERVO3_PIN for the first runout sensor.
 * For other boards you may need to define FIL_RUNOUT_PIN, FIL_RUNOUT2_PIN, etc.
 * By default the firmware assumes HIGH=FILAMENT PRESENT.
 */
//#define FILAMENT_RUNOUT_SENSOR
#if ENABLED(FILAMENT_RUNOUT_SENSOR)
  #define NUM_RUNOUT_SENSORS   1     // Number of sensors, up to one per extruder. Define a FIL_RUNOUT#_PIN for each.
  #define FIL_RUNOUT_INVERTING false // set to true to invert the logic of the sensor.
  #define FIL_RUNOUT_PULLUP          // Use internal pullup for filament runout pins.
  #define FILAMENT_RUNOUT_SCRIPT "M600"
#endif

//===========================================================================
//=============================== Bed Leveling ==============================
//===========================================================================
// @section calibrate

/**
 * Choose one of the options below to enable G29 Bed Leveling. The parameters
 * and behavior of G29 will change depending on your selection.
 *
 *  If using a Probe for Z Homing, enable Z_SAFE_HOMING also!
 *
 * - AUTO_BED_LEVELING_3POINT
 *   Probe 3 arbitrary points on the bed (that aren't collinear)
 *   You specify the XY coordinates of all 3 points.
 *   The result is a single tilted plane. Best for a flat bed.
 *
 * - AUTO_BED_LEVELING_LINEAR
 *   Probe several points in a grid.
 *   You specify the rectangle and the density of sample points.
 *   The result is a single tilted plane. Best for a flat bed.
 *
 * - AUTO_BED_LEVELING_BILINEAR
 *   Probe several points in a grid.
 *   You specify the rectangle and the density of sample points.
 *   The result is a mesh, best for large or uneven beds.
 *
 * - AUTO_BED_LEVELING_UBL (Unified Bed Leveling)
 *   A comprehensive bed leveling system combining the features and benefits
 *   of other systems. UBL also includes integrated Mesh Generation, Mesh
 *   Validation and Mesh Editing systems.
 *
 * - MESH_BED_LEVELING
 *   Probe a grid manually
 *   The result is a mesh, suitable for large or uneven beds. (See BILINEAR.)
 *   For machines without a probe, Mesh Bed Leveling provides a method to perform
 *   leveling in steps so you can manually adjust the Z height at each grid-point.
 *   With an LCD controller the process is guided step-by-step.
 */
//#define AUTO_BED_LEVELING_3POINT
//#define AUTO_BED_LEVELING_LINEAR
//#define AUTO_BED_LEVELING_BILINEAR
//#define AUTO_BED_LEVELING_UBL
#define MESH_BED_LEVELING

/**
 * Normally G28 leaves leveling disabled on completion. Enable
 * this option to have G28 restore the prior leveling state.
 */
//#define RESTORE_LEVELING_AFTER_G28

/**
 * Enable detailed logging of G28, G29, M48, etc.
 * Turn on with the command 'M111 S32'.
 * NOTE: Requires a lot of PROGMEM!
 */
//#define DEBUG_LEVELING_FEATURE

#if ENABLED(MESH_BED_LEVELING) || ENABLED(AUTO_BED_LEVELING_BILINEAR) || ENABLED(AUTO_BED_LEVELING_UBL)
  // Gradually reduce leveling correction until a set height is reached,
  // at which point movement will be level to the machine's XY plane.
  // The height can be set with M420 Z<height>
  #define ENABLE_LEVELING_FADE_HEIGHT

  // For Cartesian machines, instead of dividing moves on mesh boundaries,
  // split up moves into short segments like a Delta. This follows the
  // contours of the bed more closely than edge-to-edge straight moves.
  #define SEGMENT_LEVELED_MOVES
  #define LEVELED_SEGMENT_LENGTH 5.0 // (mm) Length of all segments (except the last one)

  /**
   * Enable the G26 Mesh Validation Pattern tool.
   */
#define G26_MESH_VALIDATION
  #if ENABLED(G26_MESH_VALIDATION)
    #define MESH_TEST_NOZZLE_SIZE    0.4  // (mm) Diameter of primary nozzle.
    #define MESH_TEST_LAYER_HEIGHT   0.2  // (mm) Default layer height for the G26 Mesh Validation Tool.
    #define MESH_TEST_HOTEND_TEMP  200.0  // (°C) Default nozzle temperature for the G26 Mesh Validation Tool.
    #define MESH_TEST_BED_TEMP      60.0  // (°C) Default bed temperature for the G26 Mesh Validation Tool.
  #endif

#endif

#if ENABLED(AUTO_BED_LEVELING_LINEAR) || ENABLED(AUTO_BED_LEVELING_BILINEAR)

  // Set the number of grid points per dimension.
  #define GRID_MAX_POINTS_X 3
  #define GRID_MAX_POINTS_Y GRID_MAX_POINTS_X

  // Set the boundaries for probing (where the probe can reach).
  #define LEFT_PROBE_BED_POSITION 25
  #define RIGHT_PROBE_BED_POSITION 181
  #define FRONT_PROBE_BED_POSITION 25
  #define BACK_PROBE_BED_POSITION 185

  // The Z probe minimum outer margin (to validate G29 parameters).
  #define MIN_PROBE_EDGE 10

  // Probe along the Y axis, advancing X after each column
  //#define PROBE_Y_FIRST

  #if ENABLED(AUTO_BED_LEVELING_BILINEAR)

    // Beyond the probed grid, continue the implied tilt?
    // Default is to maintain the height of the nearest edge.
    //#define EXTRAPOLATE_BEYOND_GRID

    //
    // Experimental Subdivision of the grid by Catmull-Rom method.
    // Synthesizes intermediate points to produce a more detailed mesh.
    //
    //#define ABL_BILINEAR_SUBDIVISION
    #if ENABLED(ABL_BILINEAR_SUBDIVISION)
      // Number of subdivisions between probe points
      #define BILINEAR_SUBDIVISIONS 3
    #endif

  #endif

#elif ENABLED(AUTO_BED_LEVELING_UBL)

  //===========================================================================
  //========================= Unified Bed Leveling ============================
  //===========================================================================

  //#define MESH_EDIT_GFX_OVERLAY   // Display a graphics overlay while editing the mesh

  #define MESH_INSET 1              // Set Mesh bounds as an inset region of the bed
  #define GRID_MAX_POINTS_X 10      // Don't use more than 15 points per axis, implementation limited.
  #define GRID_MAX_POINTS_Y GRID_MAX_POINTS_X

  #define UBL_MESH_EDIT_MOVES_Z     // Sophisticated users prefer no movement of nozzle
  #define UBL_SAVE_ACTIVE_ON_M500   // Save the currently active mesh in the current slot on M500

  //#define UBL_Z_RAISE_WHEN_OFF_MESH 2.5 // When the nozzle is off the mesh, this value is used
                                          // as the Z-Height correction value.

#elif ENABLED(MESH_BED_LEVELING)

  //===========================================================================
  //=================================== Mesh ==================================
  //===========================================================================

  #define MESH_INSET 10          // Set Mesh bounds as an inset region of the bed
  #define GRID_MAX_POINTS_X 5    // Don't use more than 7 points per axis, implementation limited.
  #define GRID_MAX_POINTS_Y GRID_MAX_POINTS_X

  //#define MESH_G28_REST_ORIGIN // After homing all axes ('G28' or 'G28 XYZ') rest Z at Z_MIN_POS

#endif // BED_LEVELING

/**
 * Points to probe for all 3-point Leveling procedures.
 * Override if the automatically selected points are inadequate.
 */
#if ENABLED(AUTO_BED_LEVELING_3POINT) || ENABLED(AUTO_BED_LEVELING_UBL)
  //#define PROBE_PT_1_X 15
  //#define PROBE_PT_1_Y 180
  //#define PROBE_PT_2_X 15
  //#define PROBE_PT_2_Y 20
  //#define PROBE_PT_3_X 170
  //#define PROBE_PT_3_Y 20
#endif

/**
 * Add a bed leveling sub-menu for ABL or MBL.
 * Include a guided procedure if manual probing is enabled.
 */
//#define LCD_BED_LEVELING

#if ENABLED(LCD_BED_LEVELING)
  #define MBL_Z_STEP 0.025    // Step size while manually probing Z axis.
  #define LCD_PROBE_Z_RANGE 4 // Z Range centered on Z_MIN_POS for LCD Z adjustment
#endif

// Add a menu item to move between bed corners for manual bed adjustment
//#define LEVEL_BED_CORNERS

#if ENABLED(LEVEL_BED_CORNERS)
  #define LEVEL_CORNERS_INSET 30    // (mm) An inset for corner leveling
  #define LEVEL_CORNERS_Z_HOP  4.0  // (mm) Move nozzle up before moving between corners
  //#define LEVEL_CENTER_TOO        // Move to the center after the last corner
#endif

/**
 * Commands to execute at the end of G29 probing.
 * Useful to retract or move the Z probe out of the way.
 */
//#define Z_PROBE_END_SCRIPT "G1 Z10 F12000\nG1 X15 Y330\nG1 Z0.5\nG1 Z10"


// @section homing

// The center of the bed is at (X=0, Y=0)
//#define BED_CENTER_AT_0_0

// Manually set the home position. Leave these undefined for automatic settings.
// For DELTA this is the top-center of the Cartesian print volume.
//#define MANUAL_X_HOME_POS 0
//#define MANUAL_Y_HOME_POS 0
//#define MANUAL_Z_HOME_POS 0

// Use "Z Safe Homing" to avoid homing with a Z probe outside the bed area.
//
// With this feature enabled:
//
// - Allow Z homing only after X and Y homing AND stepper drivers still enabled.
// - If stepper drivers time out, it will need X and Y homing again before Z homing.
// - Move the Z probe (or nozzle) to a defined XY point before Z Homing when homing all axes (G28).
// - Prevent Z homing when the Z probe is outside bed area.
//
//#define Z_SAFE_HOMING

#if ENABLED(Z_SAFE_HOMING)
  #define Z_SAFE_HOMING_X_POINT ((X_BED_SIZE) / 2)    // X point for Z homing when homing all axes (G28).
  #define Z_SAFE_HOMING_Y_POINT ((Y_BED_SIZE) / 2)    // Y point for Z homing when homing all axes (G28).
#endif

// Homing speeds (mm/m)
#define HOMING_FEEDRATE_XY (50*60)
#define HOMING_FEEDRATE_Z  (4*60)

// @section calibrate

/**
 * Bed Skew Compensation
 *
 * This feature corrects for misalignment in the XYZ axes.
 *
 * Take the following steps to get the bed skew in the XY plane:
 *  1. Print a test square (e.g., https://www.thingiverse.com/thing:2563185)
 *  2. For XY_DIAG_AC measure the diagonal A to C
 *  3. For XY_DIAG_BD measure the diagonal B to D
 *  4. For XY_SIDE_AD measure the edge A to D
 *
 * Marlin automatically computes skew factors from these measurements.
 * Skew factors may also be computed and set manually:
 *
 *  - Compute AB     : SQRT(2*AC*AC+2*BD*BD-4*AD*AD)/2
 *  - XY_SKEW_FACTOR : TAN(PI/2-ACOS((AC*AC-AB*AB-AD*AD)/(2*AB*AD)))
 *
 * If desired, follow the same procedure for XZ and YZ.
 * Use these diagrams for reference:
 *
 *    Y                     Z                     Z
 *    ^     B-------C       ^     B-------C       ^     B-------C
 *    |    /       /        |    /       /        |    /       /
 *    |   /       /         |   /       /         |   /       /
 *    |  A-------D          |  A-------D          |  A-------D
 *    +-------------->X     +-------------->X     +-------------->Y
 *     XY_SKEW_FACTOR        XZ_SKEW_FACTOR        YZ_SKEW_FACTOR
 */
//#define SKEW_CORRECTION

#if ENABLED(SKEW_CORRECTION)
  // Input all length measurements here:
  #define XY_DIAG_AC 282.8427124746
  #define XY_DIAG_BD 282.8427124746
  #define XY_SIDE_AD 200

  // Or, set the default skew factors directly here
  // to override the above measurements:
  #define XY_SKEW_FACTOR 0.0

  //#define SKEW_CORRECTION_FOR_Z
  #if ENABLED(SKEW_CORRECTION_FOR_Z)
    #define XZ_DIAG_AC 282.8427124746
    #define XZ_DIAG_BD 282.8427124746
    #define YZ_DIAG_AC 282.8427124746
    #define YZ_DIAG_BD 282.8427124746
    #define YZ_SIDE_AD 200
    #define XZ_SKEW_FACTOR 0.0
    #define YZ_SKEW_FACTOR 0.0
  #endif

  // Enable this option for M852 to set skew at runtime
  //#define SKEW_CORRECTION_GCODE
#endif

//=============================================================================
//============================= Additional Features ===========================
//=============================================================================

// @section extras

//
// EEPROM
//
// The microcontroller can store settings in the EEPROM, e.g. max velocity...
// M500 - stores parameters in EEPROM
// M501 - reads parameters from EEPROM (if you need reset them after you changed them temporarily).
// M502 - reverts to the default "factory settings".  You still need to store them in EEPROM afterwards if you want to.
//
#define EEPROM_SETTINGS // Enable for M500 and M501 commands
//#define DISABLE_M503    // Saves ~2700 bytes of PROGMEM. Disable for release!
#define EEPROM_CHITCHAT   // Give feedback on EEPROM commands. Disable to save PROGMEM.

//
// Host Keepalive
//
// When enabled Marlin will send a busy status message to the host
// every couple of seconds when it can't accept commands.
//
#define HOST_KEEPALIVE_FEATURE        // Disable this if your host doesn't like keepalive messages
#define DEFAULT_KEEPALIVE_INTERVAL 2  // Number of seconds between "busy" messages. Set with M113.
#define BUSY_WHILE_HEATING            // Some hosts require "busy" messages even during heating

//
// M100 Free Memory Watcher
//
//#define M100_FREE_MEMORY_WATCHER    // Add M100 (Free Memory Watcher) to debug memory usage

//
// G20/G21 Inch mode support
//
//#define INCH_MODE_SUPPORT

//
// M149 Set temperature units support
//
//#define TEMPERATURE_UNITS_SUPPORT

// @section temperature

// Preheat Constants
#define PREHEAT_1_TEMP_HOTEND 180
#define PREHEAT_1_TEMP_BED     70
#define PREHEAT_1_FAN_SPEED     0 // Value from 0 to 255

#define PREHEAT_2_TEMP_HOTEND 240
#define PREHEAT_2_TEMP_BED    110
#define PREHEAT_2_FAN_SPEED     0 // Value from 0 to 255

/**
 * Nozzle Park
 *
 * Park the nozzle at the given XYZ position on idle or G27.
 *
 * The "P" parameter controls the action applied to the Z axis:
 *
 *    P0  (Default) If Z is below park Z raise the nozzle.
 *    P1  Raise the nozzle always to Z-park height.
 *    P2  Raise the nozzle by Z-park amount, limited to Z_MAX_POS.
 */
#define NOZZLE_PARK_FEATURE

#if ENABLED(NOZZLE_PARK_FEATURE)
  // Specify a park position as { X, Y, Z }
  #define NOZZLE_PARK_POINT { (X_MIN_POS + 10), (Y_MAX_POS - 10), 20 }
  #define NOZZLE_PARK_XY_FEEDRATE 100   // X and Y axes feedrate in mm/s (also used for delta printers Z axis)
  #define NOZZLE_PARK_Z_FEEDRATE 5      // Z axis feedrate in mm/s (not used for delta printers)
#endif

/**
 * Clean Nozzle Feature -- EXPERIMENTAL
 *
 * Adds the G12 command to perform a nozzle cleaning process.
 *
 * Parameters:
 *   P  Pattern
 *   S  Strokes / Repetitions
 *   T  Triangles (P1 only)
 *
 * Patterns:
 *   P0  Straight line (default). This process requires a sponge type material
 *       at a fixed bed location. "S" specifies strokes (i.e. back-forth motions)
 *       between the start / end points.
 *
 *   P1  Zig-zag pattern between (X0, Y0) and (X1, Y1), "T" specifies the
 *       number of zig-zag triangles to do. "S" defines the number of strokes.
 *       Zig-zags are done in whichever is the narrower dimension.
 *       For example, "G12 P1 S1 T3" will execute:
 *
 *          --
 *         |  (X0, Y1) |     /\        /\        /\     | (X1, Y1)
 *         |           |    /  \      /  \      /  \    |
 *       A |           |   /    \    /    \    /    \   |
 *         |           |  /      \  /      \  /      \  |
 *         |  (X0, Y0) | /        \/        \/        \ | (X1, Y0)
 *          --         +--------------------------------+
 *                       |________|_________|_________|
 *                           T1        T2        T3
 *
 *   P2  Circular pattern with middle at NOZZLE_CLEAN_CIRCLE_MIDDLE.
 *       "R" specifies the radius. "S" specifies the stroke count.
 *       Before starting, the nozzle moves to NOZZLE_CLEAN_START_POINT.
 *
 *   Caveats: The ending Z should be the same as starting Z.
 * Attention: EXPERIMENTAL. G-code arguments may change.
 *
 */
//#define NOZZLE_CLEAN_FEATURE

#if ENABLED(NOZZLE_CLEAN_FEATURE)
  // Default number of pattern repetitions
  #define NOZZLE_CLEAN_STROKES  12

  // Default number of triangles
  #define NOZZLE_CLEAN_TRIANGLES  3

  // Specify positions as { X, Y, Z }
  #define NOZZLE_CLEAN_START_POINT { 30, 30, (Z_MIN_POS + 1)}
  #define NOZZLE_CLEAN_END_POINT   {100, 60, (Z_MIN_POS + 1)}

  // Circular pattern radius
  #define NOZZLE_CLEAN_CIRCLE_RADIUS 6.5
  // Circular pattern circle fragments number
  #define NOZZLE_CLEAN_CIRCLE_FN 10
  // Middle point of circle
  #define NOZZLE_CLEAN_CIRCLE_MIDDLE NOZZLE_CLEAN_START_POINT

  // Moves the nozzle to the initial position
  #define NOZZLE_CLEAN_GOBACK
#endif

/**
 * Print Job Timer
 *
 * Automatically start and stop the print job timer on M104/M109/M190.
 *
 *   M104 (hotend, no wait) - high temp = none,        low temp = stop timer
 *   M109 (hotend, wait)    - high temp = start timer, low temp = stop timer
 *   M190 (bed, wait)       - high temp = start timer, low temp = none
 *
 * The timer can also be controlled with the following commands:
 *
 *   M75 - Start the print job timer
 *   M76 - Pause the print job timer
 *   M77 - Stop the print job timer
 */
#define PRINTJOB_TIMER_AUTOSTART

/**
 * Print Counter
 *
 * Track statistical data such as:
 *
 *  - Total print jobs
 *  - Total successful print jobs
 *  - Total failed print jobs
 *  - Total time printing
 *
 * View the current statistics with M78.
 */
#define PRINTCOUNTER

//=============================================================================
//============================= LCD and SD support ============================
//=============================================================================

// @section lcd

/**
 * LCD LANGUAGE
 *
 * Select the language to display on the LCD. These languages are available:
 *
 *    en, an, bg, ca, cn, cz, cz_utf8, de, el, el-gr, es, es_utf8, eu,
 *    fi, fr, fr_utf8, gl, hr, it, kana, kana_utf8, ko_KR, nl, pl, pt,
 *    pt_utf8, pt-br, pt-br_utf8, ru, sk_utf8, tr, uk, zh_CN, zh_TW, test
 *
 * :{ 'en':'English', 'an':'Aragonese', 'bg':'Bulgarian', 'ca':'Catalan', 'cn':'Chinese', 'cz':'Czech', 'cz_utf8':'Czech (UTF8)', 'de':'German', 'el':'Greek', 'el-gr':'Greek (Greece)', 'es':'Spanish', 'es_utf8':'Spanish (UTF8)', 'eu':'Basque-Euskera', 'fi':'Finnish', 'fr':'French', 'fr_utf8':'French (UTF8)', 'gl':'Galician', 'hr':'Croatian', 'it':'Italian', 'kana':'Japanese', 'kana_utf8':'Japanese (UTF8)', 'ko_KR':'Korean', 'nl':'Dutch', 'pl':'Polish', 'pt':'Portuguese', 'pt-br':'Portuguese (Brazilian)', 'pt-br_utf8':'Portuguese (Brazilian UTF8)', 'pt_utf8':'Portuguese (UTF8)', 'ru':'Russian', 'sk_utf8':'Slovak (UTF8)', 'tr':'Turkish', 'uk':'Ukrainian', 'zh_CN':'Chinese (Simplified)', 'zh_TW':'Chinese (Taiwan)', 'test':'TEST' }
 */
//#define LCD_LANGUAGE en

/**
 * LCD Character Set
 *
 * Note: This option is NOT applicable to Graphical Displays.
 *
 * All character-based LCDs provide ASCII plus one of these
 * language extensions:
 *
 *  - JAPANESE ... the most common
 *  - WESTERN  ... with more accented characters
 *  - CYRILLIC ... for the Russian language
 *
 * To determine the language extension installed on your controller:
 *
 *  - Compile and upload with LCD_LANGUAGE set to 'test'
 *  - Click the controller to view the LCD menu
 *  - The LCD will display Japanese, Western, or Cyrillic text
 *
 * See http://marlinfw.org/docs/development/lcd_language.html
 *
 * :['JAPANESE', 'WESTERN', 'CYRILLIC']
 */
//#define DISPLAY_CHARSET_HD44780 JAPANESE

/**
 * SD CARD
 *
 * SD Card support is disabled by default. If your controller has an SD slot,
 * you must uncomment the following option or it won't work.
 *
 */
#define SDSUPPORT

/**
 * SD CARD: SPI SPEED
 *
 * Enable one of the following items for a slower SPI transfer speed.
 * This may be required to resolve "volume init" errors.
 */
//#define SPI_SPEED SPI_HALF_SPEED
//#define SPI_SPEED SPI_QUARTER_SPEED
//#define SPI_SPEED SPI_EIGHTH_SPEED

/**
 * SD CARD: ENABLE CRC
 *
 * Use CRC checks and retries on the SD communication.
 */
//#define SD_CHECK_AND_RETRY

/**
 * LCD Menu Items
 *
 * Disable all menus and only display the Status Screen, or
 * just remove some extraneous menu items to recover space.
 */
//#define NO_LCD_MENUS
//#define SLIM_LCD_MENUS

//
// ENCODER SETTINGS
//
// This option overrides the default number of encoder pulses needed to
// produce one step. Should be increased for high-resolution encoders.
//
//#define ENCODER_PULSES_PER_STEP 4

//
// Use this option to override the number of step signals required to
// move between next/prev menu items.
//
//#define ENCODER_STEPS_PER_MENU_ITEM 1

/**
 * Encoder Direction Options
 *
 * Test your encoder's behavior first with both options disabled.
 *
 *  Reversed Value Edit and Menu Nav? Enable REVERSE_ENCODER_DIRECTION.
 *  Reversed Menu Navigation only?    Enable REVERSE_MENU_DIRECTION.
 *  Reversed Value Editing only?      Enable BOTH options.
 */

//
// This option reverses the encoder direction everywhere.
//
//  Set this option if CLOCKWISE causes values to DECREASE
//
//#define REVERSE_ENCODER_DIRECTION

//
// This option reverses the encoder direction for navigating LCD menus.
//
//  If CLOCKWISE normally moves DOWN this makes it go UP.
//  If CLOCKWISE normally moves UP this makes it go DOWN.
//
//#define REVERSE_MENU_DIRECTION

//
// Individual Axis Homing
//
// Add individual axis homing items (Home X, Home Y, and Home Z) to the LCD menu.
//
//#define INDIVIDUAL_AXIS_HOMING_MENU

//
// SPEAKER/BUZZER
//
// If you have a speaker that can produce tones, enable it here.
// By default Marlin assumes you have a buzzer with a fixed frequency.
//
#define SPEAKER

//
// STARTUP CHIME
//
// Play a (non-earpiercing) startup chime on startup/serial connection
// of the Trigorilla board
//
//#define STARTUP_CHIME

//
// ENDSTOP BEEP
//
// Short 2KHz beep when endstops are hit
//
//#define ENDSTOP_BEEP

//
// The duration and frequency for the UI feedback sound.
// Set these to 0 to disable audio feedback in the LCD menus.
//
// Note: Test audio output with the G-Code:
//  M300 S<frequency Hz> P<duration ms>
//
//#define LCD_FEEDBACK_FREQUENCY_DURATION_MS 2
//#define LCD_FEEDBACK_FREQUENCY_HZ 5000

//=============================================================================
//======================== LCD / Controller Selection =========================
//========================   (Character-based LCDs)   =========================
//=============================================================================

//
// RepRapDiscount Smart Controller.
// http://reprap.org/wiki/RepRapDiscount_Smart_Controller
//
// Note: Usually sold with a white PCB.
//
//#define REPRAP_DISCOUNT_SMART_CONTROLLER

//
// ULTIMAKER Controller.
//
//#define ULTIMAKERCONTROLLER

//
// ULTIPANEL as seen on Thingiverse.
//
//#define ULTIPANEL

//
// PanelOne from T3P3 (via RAMPS 1.4 AUX2/AUX3)
// http://reprap.org/wiki/PanelOne
//
//#define PANEL_ONE

//
// GADGETS3D G3D LCD/SD Controller
// http://reprap.org/wiki/RAMPS_1.3/1.4_GADGETS3D_Shield_with_Panel
//
// Note: Usually sold with a blue PCB.
//
//#define G3D_PANEL

//
// RigidBot Panel V1.0
// http://www.inventapart.com/
//
//#define RIGIDBOT_PANEL

//
// Makeboard 3D Printer Parts 3D Printer Mini Display 1602 Mini Controller
// https://www.aliexpress.com/item/Micromake-Makeboard-3D-Printer-Parts-3D-Printer-Mini-Display-1602-Mini-Controller-Compatible-with-Ramps-1/32765887917.html
//
//#define MAKEBOARD_MINI_2_LINE_DISPLAY_1602

//
// ANET and Tronxy 20x4 Controller
//
//#define ZONESTAR_LCD            // Requires ADC_KEYPAD_PIN to be assigned to an analog pin.
                                  // This LCD is known to be susceptible to electrical interference
                                  // which scrambles the display.  Pressing any button clears it up.
                                  // This is a LCD2004 display with 5 analog buttons.

//
// Generic 16x2, 16x4, 20x2, or 20x4 character-based LCD.
//
//#define ULTRA_LCD

//=============================================================================
//======================== LCD / Controller Selection =========================
//=====================   (I2C and Shift-Register LCDs)   =====================
//=============================================================================

//
// CONTROLLER TYPE: I2C
//
// Note: These controllers require the installation of Arduino's LiquidCrystal_I2C
// library. For more info: https://github.com/kiyoshigawa/LiquidCrystal_I2C
//

//
// Elefu RA Board Control Panel
// http://www.elefu.com/index.php?route=product/product&product_id=53
//
//#define RA_CONTROL_PANEL

//
// Sainsmart (YwRobot) LCD Displays
//
// These require F.Malpartida's LiquidCrystal_I2C library
// https://bitbucket.org/fmalpartida/new-liquidcrystal/wiki/Home
//
//#define LCD_SAINSMART_I2C_1602
//#define LCD_SAINSMART_I2C_2004

//
// Generic LCM1602 LCD adapter
//
//#define LCM1602

//
// PANELOLU2 LCD with status LEDs,
// separate encoder and click inputs.
//
// Note: This controller requires Arduino's LiquidTWI2 library v1.2.3 or later.
// For more info: https://github.com/lincomatic/LiquidTWI2
//
// Note: The PANELOLU2 encoder click input can either be directly connected to
// a pin (if BTN_ENC defined to != -1) or read through I2C (when BTN_ENC == -1).
//
//#define LCD_I2C_PANELOLU2

//
// Panucatt VIKI LCD with status LEDs,
// integrated click & L/R/U/D buttons, separate encoder inputs.
//
//#define LCD_I2C_VIKI

//
// CONTROLLER TYPE: Shift register panels
//

//
// 2 wire Non-latching LCD SR from https://goo.gl/aJJ4sH
// LCD configuration: http://reprap.org/wiki/SAV_3D_LCD
//
//#define SAV_3DLCD

//=============================================================================
//=======================   LCD / Controller Selection  =======================
//=========================      (Graphical LCDs)      ========================
//=============================================================================

//
// CONTROLLER TYPE: Graphical 128x64 (DOGM)
//
// IMPORTANT: The U8glib library is required for Graphical Display!
//            https://github.com/olikraus/U8glib_Arduino
//

//
// RepRapDiscount FULL GRAPHIC Smart Controller
// http://reprap.org/wiki/RepRapDiscount_Full_Graphic_Smart_Controller
//
//#define REPRAP_DISCOUNT_FULL_GRAPHIC_SMART_CONTROLLER

//
// ReprapWorld Graphical LCD
// https://reprapworld.com/?products_details&products_id/1218
//
//#define REPRAPWORLD_GRAPHICAL_LCD

//
// Activate one of these if you have a Panucatt Devices
// Viki 2.0 or mini Viki with Graphic LCD
// http://panucatt.com
//
//#define VIKI2
//#define miniVIKI

//
// MakerLab Mini Panel with graphic
// controller and SD support - http://reprap.org/wiki/Mini_panel
//
//#define MINIPANEL

//
// MaKr3d Makr-Panel with graphic controller and SD support.
// http://reprap.org/wiki/MaKr3d_MaKrPanel
//
//#define MAKRPANEL

//
// Adafruit ST7565 Full Graphic Controller.
// https://github.com/eboston/Adafruit-ST7565-Full-Graphic-Controller/
//
//#define ELB_FULL_GRAPHIC_CONTROLLER

//
// BQ LCD Smart Controller shipped by
// default with the BQ Hephestos 2 and Witbox 2.
//
//#define BQ_LCD_SMART_CONTROLLER

//
// Cartesio UI
// http://mauk.cc/webshop/cartesio-shop/electronics/user-interface
//
//#define CARTESIO_UI

//
// LCD for Melzi Card with Graphical LCD
//
//#define LCD_FOR_MELZI

//
// SSD1306 OLED full graphics generic display
//
//#define U8GLIB_SSD1306

//
// SAV OLEd LCD module support using either SSD1306 or SH1106 based LCD modules
//
//#define SAV_3DGLCD
#if ENABLED(SAV_3DGLCD)
  //#define U8GLIB_SSD1306
  #define U8GLIB_SH1106
#endif

//
// Original Ulticontroller from Ultimaker 2 printer with SSD1309 I2C display and encoder
// https://github.com/Ultimaker/Ultimaker2/tree/master/1249_Ulticontroller_Board_(x1)
//
//#define ULTI_CONTROLLER

//
// TinyBoy2 128x64 OLED / Encoder Panel
//
//#define OLED_PANEL_TINYBOY2

//
// MKS MINI12864 with graphic controller and SD support
// http://reprap.org/wiki/MKS_MINI_12864
//
//#define MKS_MINI_12864

//
// Factory display for Creality CR-10
// https://www.aliexpress.com/item/Universal-LCD-12864-3D-Printer-Display-Screen-With-Encoder-For-CR-10-CR-7-Model/32833148327.html
//
// This is RAMPS-compatible using a single 10-pin connector.
// (For CR-10 owners who want to replace the Melzi Creality board but retain the display)
//
//#define CR10_STOCKDISPLAY

//
// ANET and Tronxy Graphical Controller
//
//#define ANET_FULL_GRAPHICS_LCD  // Anet 128x64 full graphics lcd with rotary encoder as used on Anet A6
                                  // A clone of the RepRapDiscount full graphics display but with
                                  // different pins/wiring (see pins_ANET_10.h).

//
// MKS OLED 1.3" 128 × 64 FULL GRAPHICS CONTROLLER
// http://reprap.org/wiki/MKS_12864OLED
//
// Tiny, but very sharp OLED display
//
//#define MKS_12864OLED          // Uses the SH1106 controller (default)
//#define MKS_12864OLED_SSD1306  // Uses the SSD1306 controller

//
// Silvergate GLCD controller
// http://github.com/android444/Silvergate
//
//#define SILVER_GATE_GLCD_CONTROLLER

//=============================================================================
//============================  Other Controllers  ============================
//=============================================================================

//
// CONTROLLER TYPE: Standalone / Serial
//

//
// LCD for Malyan M200 printers.
// This requires SDSUPPORT to be enabled
//
//#define MALYAN_LCD

//
// CONTROLLER TYPE: Keypad / Add-on
//

//
// RepRapWorld REPRAPWORLD_KEYPAD v1.1
// http://reprapworld.com/?products_details&products_id=202&cPath=1591_1626
//
// REPRAPWORLD_KEYPAD_MOVE_STEP sets how much should the robot move when a key
// is pressed, a value of 10.0 means 10mm per click.
//
//#define REPRAPWORLD_KEYPAD
//#define REPRAPWORLD_KEYPAD_MOVE_STEP 10.0

//=============================================================================
//=============================== Extra Features ==============================
//=============================================================================

// @section extras

// Increase the FAN PWM frequency. Removes the PWM noise but increases heating in the FET/Arduino
//#define FAST_PWM_FAN

// Use software PWM to drive the fan, as for the heaters. This uses a very low frequency
// which is not as annoying as with the hardware PWM. On the other hand, if this frequency
// is too low, you should also increment SOFT_PWM_SCALE.
//#define FAN_SOFT_PWM

// Incrementing this by 1 will double the software PWM frequency,
// affecting heaters, and the fan if FAN_SOFT_PWM is enabled.
// However, control resolution will be halved for each increment;
// at zero value, there are 128 effective control positions.
#define SOFT_PWM_SCALE 0

// If SOFT_PWM_SCALE is set to a value higher than 0, dithering can
// be used to mitigate the associated resolution loss. If enabled,
// some of the PWM cycles are stretched so on average the desired
// duty cycle is attained.
//#define SOFT_PWM_DITHER

// Temperature status LEDs that display the hotend and bed temperature.
// If all hotends, bed temperature, and target temperature are under 54C
// then the BLUE led is on. Otherwise the RED led is on. (1C hysteresis)
//#define TEMP_STAT_LEDS

// M240  Triggers a camera by emulating a Canon RC-1 Remote
// Data from: http://www.doc-diy.net/photo/rc-1_hacked/
//#define PHOTOGRAPH_PIN     23

// SkeinForge sends the wrong arc g-codes when using Arc Point as fillet procedure
//#define SF_ARC_FIX

// Support for the BariCUDA Paste Extruder
//#define BARICUDA

// Support for BlinkM/CyzRgb
//#define BLINKM

// Support for PCA9632 PWM LED driver
//#define PCA9632

/**
 * RGB LED / LED Strip Control
 *
 * Enable support for an RGB LED connected to 5V digital pins, or
 * an RGB Strip connected to MOSFETs controlled by digital pins.
 *
 * Adds the M150 command to set the LED (or LED strip) color.
 * If pins are PWM capable (e.g., 4, 5, 6, 11) then a range of
 * luminance values can be set from 0 to 255.
 * For Neopixel LED an overall brightness parameter is also available.
 *
 * *** CAUTION ***
 *  LED Strips require a MOSFET Chip between PWM lines and LEDs,
 *  as the Arduino cannot handle the current the LEDs will require.
 *  Failure to follow this precaution can destroy your Arduino!
 *  NOTE: A separate 5V power supply is required! The Neopixel LED needs
 *  more current than the Arduino 5V linear regulator can produce.
 * *** CAUTION ***
 *
 * LED Type. Enable only one of the following two options.
 *
 */
//#define RGB_LED
//#define RGBW_LED

#if ENABLED(RGB_LED) || ENABLED(RGBW_LED)
  #define RGB_LED_R_PIN 34
  #define RGB_LED_G_PIN 43
  #define RGB_LED_B_PIN 35
  #define RGB_LED_W_PIN -1
#endif

// Support for Adafruit Neopixel LED driver
//#define NEOPIXEL_LED
#if ENABLED(NEOPIXEL_LED)
  #define NEOPIXEL_TYPE   NEO_GRBW // NEO_GRBW / NEO_GRB - four/three channel driver type (defined in Adafruit_NeoPixel.h)
  #define NEOPIXEL_PIN    4        // LED driving pin on motherboard 4 => D4 (EXP2-5 on Printrboard) / 30 => PC7 (EXP3-13 on Rumba)
  #define NEOPIXEL_PIXELS 30       // Number of LEDs in the strip
  #define NEOPIXEL_IS_SEQUENTIAL   // Sequential display for temperature change - LED by LED. Disable to change all LEDs at once.
  #define NEOPIXEL_BRIGHTNESS 127  // Initial brightness (0-255)
  //#define NEOPIXEL_STARTUP_TEST  // Cycle through colors at startup
#endif

/**
 * Printer Event LEDs
 *
 * During printing, the LEDs will reflect the printer status:
 *
 *  - Gradually change from blue to violet as the heated bed gets to target temp
 *  - Gradually change from violet to red as the hotend gets to temperature
 *  - Change to white to illuminate work surface
 *  - Change to green once print has finished
 *  - Turn off after the print has finished and the user has pushed a button
 */
#if ENABLED(BLINKM) || ENABLED(RGB_LED) || ENABLED(RGBW_LED) || ENABLED(PCA9632) || ENABLED(NEOPIXEL_LED)
  #define PRINTER_EVENT_LEDS
#endif

/**
 * R/C SERVO support
 * Sponsored by TrinityLabs, Reworked by codexmas
 */

/**
 * Number of servos
 *
 * For some servo-related options NUM_SERVOS will be set automatically.
 * Set this manually if there are extra servos needing manual control.
 * Leave undefined or set to 0 to entirely disable the servo subsystem.
 */
//#define NUM_SERVOS 3 // Servo index starts with 0 for M280 command

// Delay (in milliseconds) before the next move will start, to give the servo time to reach its target angle.
// 300ms is a good value but you can try less delay.
// If the servo can't reach the requested position, increase it.
#define SERVO_DELAY { 300 }

// Only power servos during movement, otherwise leave off to prevent jitter
//#define DEACTIVATE_SERVOS_AFTER_MOVE

/**
 * Select your version of the Trigorilla (RAMPS1.4) board here.
 *
 * 0 = Default Trigorilla
 * 1 = Newer Trigorilla v1.1 (first seen late 2018)
 *
 * The only major difference is a slight change on the servo pin mapping.
 * This setting only is relevant if you want to use BLtouch or similar
 * mods to be used via servo pins.
 * The new version is to be identified by a "TRIGORILLA1.1" lettering
 * on the upper left of the PCB silkscreen.
 */
#define TRIGORILLA_VERSION 0

// Enable Anycubic TFT
#define ANYCUBIC_TFT_MODEL
#define ANYCUBIC_FILAMENT_RUNOUT_SENSOR
//#define ANYCUBIC_TFT_DEBUG

#endif // CONFIGURATION_H

 \end{lstlisting}


