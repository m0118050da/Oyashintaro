\chapter{開発したプリンターの精度調査実験}
\label{chp:first}

\section{プリンターの動作検証}
\label{sec:paragraph}
始めに,開発した氷をマテリアルとした3Dプリンターの動作テストを2つに分けて行った.
1つ目が,プリンターの動作テストだ.2つ目が,氷を積層できるかの調査を行った.
この二つの検証をこなうことにより,プリンターの挙動や造形の特徴を知ることができる。

\section{プリンター動作の確認}
\label{sec:paragraph}
開発した氷をマテリアルとした3Dプリンターの動作テストを行った.
一般に販売されているFDM方式の3Dプリンター「 Anycubic i3 Mega S 」を改造して使用している.
ノズルを通常の取り付け位置に設置してしまうと,ノズルの上部が3Dプリンターの上部と接触してしまうため,今回はノズルを左右を繋ぐポールの後ろ側に設置し,そのうえで,ポールから1.5㎝離している.
そのため, Ultimaker Cura で表示されている,印刷が可能な位置と比べ5cm後ろにずれている.
また,開発した氷をマテリアルとした3Dプリンターの設計上アルミプレートの上でしか造形ができない.
そのため,実際に造形できる範囲は 100mm × 80mm × 250mm であることが分かった.

シリンダーの押し出し機構については,基本的には問題なく作動した.
しかし,3Dプリンターが作動していないタイミングでも,ノズルの先から水が漏れ出してしまうのが問題として浮上した.
また,おそらく水が漏れ出したことに起因して,シリンダー内部の圧力が低下した.この影響により印刷を始めた際にしばらくノズルから水が出ない問題が発生することが分かった.
しかし,現状 Ultimaker Cura の設定により、始めにシリンダーを少し押し出す設定になっている.これにより印刷には影響していないが,対処的な解決策であり,根本的な解決が必要だ.


\section{氷造形の初期実験}
\label{sec:paragraph}
水が綺麗な線が引けて凍るか,積層されるか,開発した氷をマテリアルとした3Dプリンターが実際に氷の造形ができるかをテストした.
線を引く実験では,水と砂糖の割合が1:1のものを使い,印刷速度,押し出し量,ライン幅,の調整を行った.
純粋な水に比べ、砂糖を混ぜた水は凍るまでの速度が遅く,速度を上げすぎると,凍る前に次の層の造形が始まってしまう一方,遅すぎると,ノズルと造形物が凍ってしまい造形ができなくなることが分かった.
押し出し量,ライン幅,については,ノズルの太さ(1mm)を踏まえたうえで,調整しないと,スカスカの造形または,水が飛び出したような造形になってしまう事が分かった.
この時の最適なパラメーターは, ライン幅:3.0mm 印刷速度:50.0mmということが分かった.
このパラメータを基に素材ごとの積層の実験を行った.

\section{氷の積層実験}
\label{sec:paragraph}
氷造形の初期実験で




初期実験から Slic3r のパラメータを次のように設定した.ノズルの移動速度のパラメータであ
3 る Speed を 50mm/s に設定し,次のレイヤーを造形する際ノズルをどれだけ上げるかを決めるパ
4 ラメータの Layer height を 0.1mm,外壁の厚みの層数を設定するパラメターの Perimeters を 3
5 に変更した.
6 この設定で図 7.1 のような GCode を生成した.高さが 5mm に設定されており,5mm の高さ
7 まで積層できるかを調査した.
8 造形の際に,水の押し出し量が少なすぎると,特定の場所だけ積層され一部が全く造形されな
9 いと言う結果になった.水の押し出し量は,フィラメントの押し出し量を設定する,Extrusion
10 multiplier のパラメータで制御できる.初期は 0.1 で始め 0.1 ずつ足していき造形した.結果
11 Extrusion multiplier が 0.4 の時にうまく積層ができ,図 7.2 のような氷を造形することができ
12 た.飛び出している部分は最初の造形の際に生まれるもので,今回は無視する.計測の結果,高
13 さ 5mm,幅 4mm と言う結果になった.
14 印刷中にヘッドパーツが造形中の氷に干渉する場面があったがヘッドが加熱されていたため,
15 溶かしながら進み造形の失敗を防いでいた.また,押し出される水は少量でアルミトレーや氷に
16 表面張力で吸い付き,凍ることで積層されていることがわかった.ヘッドの上げ率や温度を調節
17 することで,さらに精度や効率を上げることができる可能性がある.
18 押し出し量が多すぎると,図 7.3 のようにところどころふくらみのある形状になってしまう.



