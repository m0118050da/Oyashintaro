\chapter{開発したプリンターの精度調査実験}
\label{chp:first}

\section{プリンターの動作検証}
\label{sec:paragraph}
始めに,開発した氷をマテリアルとした3Dプリンターの動作テストを2つに分けて行った.
1つ目が,プリンターの動作テストだ.2つ目が,氷を積層できるかの調査を行った.
この二つの検証をこなうことにより,プリンターの挙動や造形の特徴を知ることができる。

\section{プリンター動作の確認}
\label{sec:paragraph}
開発した氷をマテリアルとした3Dプリンターの動作テストを行った.
一般に販売されているFDM方式の3Dプリンター「 Anycubic i3 Mega S 」を改造して使用している.
ノズルを通常の取り付け位置に設置してしまうと,ノズルの上部が3Dプリンターの上部と接触してしまうため,今回はノズルを左右を繋ぐポールの後ろ側に設置し,そのうえで,ポールから1.5㎝離している.
そのため,CURAで表示されている,印刷が可能な位置と比べ5cm後ろにずれている.
また,開発した氷をマテリアルとした3Dプリンターの設計上アルミプレートの上でしか造形ができない.
そのため,実際に造形できる範囲は 100mm × 80mm × 250mm であることが分かった.

シリンダーの押し出し機構については,基本的には問題なく作動した.
しかし,3Dプリンターが作動していないタイミングでも,ノズルの先から水が漏れ出してしまうのが問題として浮上した.
また,おそらく水が漏れ出したことに起因して,シリンダー内部の圧力が低下した.この影響により印刷を始めた際にしばらくノズルから水が出ない問題が発生することが分かった.
しかし,現状CURAの設定により、始めにシリンダーを少し押し出す設定になっている.これにより印刷には影響していないが,対処的な解決策であり,根本的な解決が必要だ.


\section{氷をマテリアルとした造形}
\label{sec:paragraph}
開発した氷をマテリアルとした3Dプリンターが実際に氷の造形ができるかをテストした.

21q


ノズルから供給された水が綺麗なラインを引いて凍るか,積層することが可能かどうかを調べ
17 た.ラインを引く実験ではノズルの移動速度が速すぎると,凍る前に次の造形が始まってしまい,
18 直線状に凍らない問題があった.そこで,Slic3r のパラメータを変更し調査を行った.ノズルの
19 移動速度の速度のパラメータである Speed を変更し,50mm/s にすることで造形できることがわ
20 かった.この時,氷の幅は 3mm,高さが 0.1mm という結果になったため,このパラメータを元
21 に積層の実験を行った.


\section{氷の積層実験}
\label{sec:paragraph}




