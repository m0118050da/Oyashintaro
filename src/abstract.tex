新たな3Dプリンターの素材を開発することは,多くの人が3Dプリンターを使い新たなプロダクトを作り出し,人類の想像力を最大化させる上で重要なことだと考える.
数ある3Dプリンターの素材の中から,私は氷を素材として造形ができる3Dプリンターの開発を行った.

通常の3Dプリンターと近い操作方法で使用ができ,常温で造形が可能かつ,リアルタイムで造形が可能な氷をマテリアルとした3Dプリンターの提案をする.

3Dプリンターのスライサーソフトとして世界中で使用されるスライサーソフトウェアである Ultimaker Cura で操作が可能な氷をマテリアルとした3Dプリンターの開発を行った.
開発した3Dプリンターは,液体窒素で造形用ベッドを冷やすことで常温の環境下でリアルタイムに造形を行う.
また,通常の水と粘性を持たせた水ではどちらの方が氷の造形に適しているのか調査を行った.
粘性を持たせた水を使用すため,シリンダーとノズルとの距離が遠くなると,抵抗で粘性を持たせた水が通りにくくなるため,
シリンダーとノズルが直結しそこから水を絞りだす機構を開発した.

実験では,液体窒素を使い常温の環境下でリアルタイムに造形ができるかの調査と純粋な水を使用した場合と水と砂糖を1:3で混ぜた液体でどちらの方が氷の造形に適しているのか調査を行った.
氷をマテリアルとした3Dプリンターでの造形を行う際,水と砂糖を混ぜた粘度を上げた水の方が,適しているということが分かった.
また,スライサーソフトUltimaker Curaを利用して造形できるため,3Dプリンターを扱えるもの方なら,少ない学習コストで使用することができる.
よって,通常の3Dプリンターと近い操作方法で使用ができ,常温で造形が可能かつ,リアルタイムで造形が可能な氷をマテリアルとした3Dプリンターを開発できた.
