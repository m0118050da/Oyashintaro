新たな3Dプリンターの素材を開発することは,多くの人が3Dプリンターを使い新しいものを作り出し、人類の想像力を最大化させるうえで重要なことだと考える.
その中でも,私は氷をマテリアルとした3Dプリンターの開発を行おうと考えた.

氷の彫刻は,世界中で様々なイベントやアート作品に用いられ,多くの人々に親しまれて様々な作品が作られている.
しかし,氷の作品は作るのに時間がかかり,彫刻の技術や設備が必要となるため,誰でも簡単に触れ合えるものではない.
そのため,一般の人でも扱いが可能かつ,一般の3Dプリンターと同程度の速度とある程度の精度を両立した氷をマテリアルとした3Dプリンターの提案する.

水の粘度を上げることで造形精度を向上さ,
世界中で使用されている3Dプリントおよびスライサーソフトウェアである Ultimaker Cura で操作が可能な氷をマテリアルとした3Dプリンターの開発を行った.
開発したプリンターは,液体窒素でベッドを冷やしながら印刷を行う.
シリンダーとノズルとの距離が遠くなると,抵抗で粘性を持たせた水が通りにくくなるため,シリンダーとノズルが直結しそこから液体を絞りだす機構を開発した.


実験では,純粋な水を使用した場合と水と砂糖を1:3で混ぜた液体での場合で比較を行った.
氷をマテリアルとした3Dプリンターでの造形を行う際,水と砂糖を混ぜた液体の方が,適しているということが分かった.
また,一般で広く使われているスライサーソフトUltimaker Curaを利用して造形できるため,3Dプリンターを扱えるもの方なら,少ない学習コストで使用することができる.
よって,一般の人でも扱いが可能かつ,一般のユーザーが設計したデータにできるだけ近い形に印刷できる氷をマテリアルとした3Dプリンターの開発ができたと考える.

しかし,今回開発した氷をマテリアルとした3Dプリンターにもいくつか課題が残る.
1つ目が予備実験において,手で試した際の精度を自動かさせた際に再現できていないこと.
2つ目一度に造形できる量に制限があること.
この二つを解決できたらと考える.

一般の人が気軽に氷の造形を楽しみ,様々な氷の造形物を創造するには,ノズルサイズの調整や水の温度の調整,粘度の調整なの検証を行い,さらなる精度と速度の向上が必要となる.



