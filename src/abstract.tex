氷の造形物は彫刻で作るのが一般的だが,スキルや材料の調達などの問題で誰もが簡単に製作でき
る訳ではない.本稿では,先行研究である「氷をマテリアルとしてた3 D プリンターの開発 著者:東 京
工 科 大 学 大 学 院 藤田 大樹」の液体窒素を使い FDM 方式で氷の造形を行うプリンタを中心に問題点
の改良を行う.特に問題として抱えていた、造形速度と造形精度の向上とオーバーハングの実現を目指す。
その手法として、水の粘度を変える方法と造形物のサポートの充填率のパラメーターを変えながら印刷し
関係性を調査し,その結果から,氷の造形物を印刷するのに適したパラメータを発見する.これにより,
ユーザーは特別な知識がなくても氷プリンター用の GCode を作ることができる.氷の造形物は,時間の
経過で溶けて完全に消失する性質を持っており,これは 3D プリンターに新しい表現を与える.
