\chapter{まとめ}
\label{chp:first}

近年3Dプリンターの低価格化が進んだことで一般にも普及が進んでいる.
これにより,これまで生産者と消費者は別の者であったが,生産者と消費者が同一の存在となるなりつつあるのだ.
消費者の生産者化により,これまでにない発想の商品が数多く登場し,より便利なこれまでにない発想の商品はデジタル社会により,世界中に拡散され,人類社会の発展に貢献される.
3Dプリンターは人類の可能性を最大化させるためのツールでもある.その3Dプリンターは印刷できる素材が限られているのが現状である.
新たな3Dプリンターの素材を開発することは,多くの人が3Dプリンターを使い新しいものを作り出し、人類の想像力を最大化させるうえで重要なことだと考えた.
その中でも,私は氷をマテリアルとした3Dプリンターの開発を行おうと考えた.

氷の彫刻は,世界中で様々なイベントやアート作品に用いられ,多くの人々に親しまれて様々な作品が作られている.
しかし,氷の作品は作るのに時間がかかり,彫刻の技術や設備が必要となるため,誰でも簡単に触れ合えるものではない.
また、現在開発されている氷をマテリアルとした3Dプリンターはマグカップサイズのものを作るのに50時間ほどかかるもなど特殊な環境や知識が必要なものしかない.
そのため,一般の人でも扱いが可能かつ,一般の3Dプリンターと同程度の速度とある程度の精度を両立した氷をマテリアルとした3Dプリンターの提案する.



それぞれの要素について実装にするにあたり、以上のことが有効ではないかと考える.
「ある程度の精度で造形ができること.」「通常の3Dプリンターと同程度の速度で印刷ができること.」を満たすために液体窒素を使った造形方法が有効ではないかと考える.
また,他の研究では,特殊な機材を使用し,装置が高価になりがちである.液体窒素は,日本各地で手に入る上,価格も1リットルあたり300円と安価であるため,今回の研究で使用することにした.
「3Dプリンターが扱える人であれば,短時間で扱えるようになること.」を満たすためには,既存の3Dプリンターと同様の使用方法で使える必要があるため,世界中で使用されている3Dプリントおよびスライサーソフトウェアである Ultimaker Cura で操作が可能である必要があると考える.
氷の定義をしたことで、純粋な水以外でも氷の造形ができる.純粋な水を積層する場合,水の粘度が低いため,固まる前に広がってしまう.そのため造形精度が悪く,造形物のからはみ出した部分には造形ができず,オーバーハングなども造形することが難しい.
よって,水の粘度を上げることにより,上記の問題の解決や精度を向上させることができるのではと考える.
また,粘度を上げる手段として,いくつかの方法が考えられる.水に砂糖などを加え粘度を上げる方法とシャーベット状のものをマテリアルとして使用する方法だ.
シャーベット状のものを使用する場合は,温度管理が必要になるため,今回の機構では,砂糖を加え粘度を高めたものをマテリアルとして使用する.







