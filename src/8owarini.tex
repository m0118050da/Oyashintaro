\chapter{まとめ}
\label{chp:first}

近年3Dプリンターの低価格化が進んだことで一般にも普及が進んでいる.
これにより,これまで生産者と消費者は別の者であったが,生産者と消費者が同一の存在となるなりつつあるのだ.
消費者の生産者化により,これまでにない発想の商品が数多く登場し,より便利なこれまでにない発想の商品はデジタル社会により,世界中に拡散され,人類社会の発展に貢献される.
3Dプリンターは人類の可能性を最大化させるためのツールである.その3Dプリンターは印刷できる素材が限られているのが現状である.
新たな3Dプリンターの素材を開発することは,多くの人が3Dプリンターを使い新しいものを作り出し、人類の想像力を最大化させるうえで重要なことだと考える.
その中でも,私は氷をマテリアルとした3Dプリンターの開発を行った.

氷の彫刻は,世界中で様々なイベントやアート作品に用いられ,多くの人々に親しまれて様々な作品が作られている.
しかし,氷の作品は作るのに時間がかかり,彫刻の技術や設備が必要となるため,誰でも簡単に触れ合えるものではない.
また、現在開発されている氷をマテリアルとした3Dプリンターはマグカップサイズのものを作るのに50時間ほどかかるもなど特殊な環境や知識が必要である.
そのため,通常の3D プリンターと近い操作方法で使用ができ,常温で造形が可能かつ,リアルタイムで造形が可能な氷をマテリアルとした3Dプリンターの提案をする.

氷をマテリアルとした3Dプリンターを開発するにあたり,
純粋な水を積層すると,水の粘度が低いため,固まる前に広がってしまうため,造形精度が悪く,オーバーハングなども造形することが難しい.
よって,水の粘度を上げることにより,上記の問題の解決や精度を向上させることができるのではと考えた.
また,他の研究では,特殊な機材を使用し,装置が高価になりがちである.
日本各地で手に入り,価格も1リットルあたり300円と安価であるため,液体窒素を冷却材として使用する.
最後に通常の3D プリンターと近い操作方法で使用ができるようにするために,広く使われている3Dプリントのスライサーソフトウェアである Ultimaker Cura で操作が可能である必要があると考えると考え,
氷をマテリアルとした3Dプリンターの開発を行った.
開発したプリンターは,液体窒素でベッドを冷やしながら印刷を行う.
シリンダーとノズルとの距離が遠くなると,抵抗で粘性を持たせた水が通りにくくなるため,シリンダーとノズルが直結しそこから液体を絞りだす機構を開発した.


実験では,純粋な水を使用した場合と水と砂糖を1:3で混ぜた液体での場合で比較を行った.
水単体と比べ水と砂糖を混ぜた液体での方が造形精度が高く,積層もスムーズに行われた.
氷をマテリアルとした3Dプリンターでの造形を行う際,水と砂糖を混ぜた液体の方が,適しているということが分かった.
また,開発において使用したものは既存の3Dプリンターの部品やどこでも手に入れやすいものを多く使用した.
特に,一般で広く使われているスライサーソフトUltimaker Curaを利用して造形できるため,3Dプリンターを扱えるもの方なら,少ない学習コストで使用することができる.
よって,通常の3D プリンターと近い操作方法で使用ができ,常温で造形が可能かつ,リアルタイムで造形が可能な氷をマテリアルとした3Dプリンターの開発ができた.

しかし,今回開発した氷をマテリアルとした3Dプリンターにもいくつか課題が残る.
1つ目が予備実験において,手で試した際の精度を自動かさせた際に再現できていないこと.
2つ目一度に造形できる量に制限があること.
この二つを解決できたらと考える.

今回の研究開発では,氷をマテリアルとした造形物をある程度の速度と精度で制作を行える機構を開発し, Ultimaker Cura で操作可能なプログラムを組むことで既存の氷をマテリアルとした3Dプリンターに比べ,一般の人でも扱いが可能な3Dプリンターになった
水に粘性を持たせることで,通常の3D プリンターと近い操作方法で使用ができ,常温で造形が可能かつ,リアルタイムで造形が可能な氷をマテリアルとした3Dプリンターの開発ができた.
しかし,現状のプリンターではまだ,オーバーハングが不十分なこと造形できる量に制限があることなどいくつか問題を抱えている.
また,ノズルのサイズの調整や水の温度の調整,粘度の調整などまだまだ,ある程度の速度と精度をもった3Dプリンターの開発において,検証可能な要素が残されている.
一般の人が気軽に氷の造形を楽しみ,様々な氷の造形物を創造するには,上記の要素の検証を行い,さらなる精度と速度の向上が必要となる.









